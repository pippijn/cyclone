\section{Installing Cyclone}
\label{sec:install}
Cyclone currently only runs on 32-bit machines.  It has been tested on
Linux, Windows 98/NT/2K/XP using the Cygwin environment, and on Mac OS
X\@.  Other platforms might or might not work.  Right now, there are a
few 32-bit dependencies in the compiler, so the system will probably
not work on a 64-bit machine without major changes.

To install and use Cyclone, you'll need to use the Gnu utilities,
including gcc (the Gnu C compiler) and Gnu-Make.  For Windows, you
should first install the latest version of the
\href{http://cygwin.com/}{Cygwin} utilities to do the build, and make
sure that the Cygwin \texttt{bin} directory is on your path. We use
some features of gcc extensively, so Cyclone definitely will not build
with another C compiler.

Cyclone is distributed as a compressed archive (a .tar.gz file).
Unpack the distribution into a directory; if you are installing
Cyclone on a Windows system, we suggest you choose
\texttt{c:/cyclone}.

From here, follow the instructions in the INSTALL file included in the
distribution.

% In any case, in these
% instructions we'll use \textit{cycdir} to refer to the directory.

% Next, enter the directory (\texttt{cd} \textit{cycdir}) and type
% ``\texttt{make}.''  This will create executables for the Cyclone
% compiler (\texttt{cyclone} in Unix systems, or \texttt{cyclone.exe} in
% Windows systems) and some utilities in the \texttt{bin} subdirectory.

% Finally, add \textit{cycdir}\texttt{/bin} to your \texttt{PATH}.  In
% addition, if \textit{cycdir} is not \texttt{c:/cyclone}, you must set
% the environment variable \texttt{CYCLONE_INCLUDE_PATH} to
% \textit{cycdir}\texttt{/include}, and the variable
% \texttt{CYCLONE_EXEC_PREFIX} to \textit{cycdir}\texttt{/bin/cyc-lib}.

% You should now be able to compile Cyclone programs.

% \subsection*{Bootstrapping}

% If you're interested in working on the compiler itself, you'll want to
% know more about how the sources are organized.  The Cyclone directory
% has the following subdirectories:
% \begin{center}
% \begin{tabular}{ll}
% \texttt{bin/}        & executables\\
% ~~\texttt{genfiles/} & generated C files for making the executables\\
% \texttt{doc/}        & documentation\\
% \texttt{include/}    & .h files for the libraries\\
% \texttt{lib/}        & Cyclone source code for the libraries\\
% \texttt{src/}        & Cyclone source code for the compiler\\
% \texttt{tests/}      & some test code written in Cyclone\\
% \texttt{tools/}      & source code of the lexer and parser generators
% \end{tabular}
% \end{center}

% The Cyclone compiler is written in Cyclone, but since you won't have a
% Cyclone compiler to start with, we have a directory,
% \texttt{bin/genfiles}, that contains C files generated from our
% Cyclone source files.  During the installation, these are compiled by
% GCC to produce the executables in \texttt{bin}.  Once this is done,
% the resulting Cyclone compiler can be used to compile the compiler
% written in Cyclone.

% To do this, simply type ``\texttt{make cyclone_src}.''  This will use
% the Cyclone compiler to compile all of the files in the \texttt{lib}
% and \texttt{src} directories (it will not overwrite the binaries in
% the \texttt{bin/} directory).

% If you want to make a change to the compiler and then start using the
% change, you can type ``\texttt{make update}'' which will update the
% \texttt{genfiles/} with any different C files that you generated.
% Then you can type ``\texttt{make cyclone}'' to bootstrap with the new
% code.

% Local Variables:
% TeX-master: "main-screen"
% End:
