\section{Libraries}
%HEVEA \newcommand{\parbox}[3][x]{#3}
\newcommand{\code}[1]{\texttt{#1}}
\newcommand{\var}[1]{\texttt{#1}}
\newcommand{\vvar}[1]{{\tt\textbf{\textit{#1}}}}
\newcommand{\defunlabel}[1]{%
  \parbox[b]{\labelwidth}{\makebox[0pt][l]{#1}\\}}
\newenvironment{defun}[2]{%
  \begin{list}{}{}%
    \renewcommand{\makelabel}{\defunlabel}%
    \index{#1@\texttt{#1}}%
  \item[\textbf{\texttt{#1}}\texttt{#2}]}{\end{list}}
\newenvironment{defun2}[3]{%
  \begin{list}{}{}%
    \renewcommand{\makelabel}{\defunlabel}%
    \index{#2@\texttt{#2()}}%
  \item[{\tt {#1}~\textbf{#2}{#3}}]}{\end{list}}
\newenvironment{deexn}[2]{%
  \begin{list}{}{}%
    \renewcommand{\makelabel}{\defunlabel}%
    \index{#1@\texttt{#1}}%
  \item[{\tt xtunion exn~\lb~extern~\textbf{#1}{#2}~\rb}]}{\end{list}}
%%% \newenvironment{defun}[2]{\noindent\index{#1@\texttt{#1}}
%%%   \textbf{\texttt{#1}}\texttt{#2}\begin{quote}}{\end{quote}}
\newcommand{\tindex}[1]{}
\newcommand{\exindex}[1]{}

\subsection{Array}
\begin{verbatim}
#include <array.h>
using Array;
\end{verbatim}

\subsubsection*{Exceptions}

\begin{deexn}{Array_mismatch}{}
  Thrown when two arrays don't have the same size.
\end{deexn}


\subsubsection*{Basic operations}

\begin{defun2}{`a ?}{copy}{(`a::B {?}`r \vvar{x});}
  Return a fresh copy of the array \vvar{x}, allocated on the heap.
\end{defun2}

\begin{defun2}{`a ?}{extract}{(`a::B {?}`r1 \vvar{x},int \vvar{start},int *`r2 \vvar{n_opt});}
  Create a new array on the heap, with elements extracted from the
  array \vvar{x}, beginning with element \vvar{start}.  If
  \vvar{n_opt} is \texttt{null}, all of the elements from \vvar{start}
  to the end of \vvar{x} are extracted; if \vvar{n_opt} points to an
  integer $n$, then $n$ elements are extracted.  If $n\leq 0$ or there
  are less than $n$ elements in \vvar{x} starting at \vvar{start},
  then \texttt{Core::InvalidArg("Array::extract")} is thrown.
\end{defun2}

\subsubsection*{Iteration and mapping}

\begin{defun2}{`b ?}{map}{(`b \vvar{f}(`a),`a {?}`r \vvar{x});}
  Heap-allocate an array whose elements are obtained by applying the
  function \vvar{f} to each element of \vvar{x} in turn.
\end{defun2}

\begin{defun2}{`b ?}{map_c}{(`b \vvar{f}(`c,`a),`c \vvar{env},`a {?}`r \vvar{x});}
  A version of \code{map} where the function \vvar{f} requires an
  environment.
\end{defun2}

\begin{defun2}{void}{imp_map}{(`a \vvar{f}(`a),`a {?}`r \vvar{x});}
  An imperative version of \code{map}: each element of \vvar{x} is
  replaced with the result of applying \vvar{f} to the element.
\end{defun2}

\begin{defun2}{void}{imp_map_c}{(`a \vvar{f}(`b,`a),`b \vvar{env},`a {?}`r \vvar{x});}
  A version of \code{imp_map} where the function \vvar{f} requires an
  environment.
\end{defun2}

\begin{defun2}{`c ?}{map2}{(`c \vvar{f}(`a,`b),`a {?}`r1 \vvar{x},`b {?}`r2 \vvar{y});}
  If \vvar{x} and \vvar{y} both have size $n$, then \code{map2}
  heap-allocates a new array where the $i^{\rm th}$ element, for $i$
  between 0 and $n$, is \code{f(x[$i$],y[$i$])}.
    
  \code{Array_mismatch} is thrown if \vvar{x} and \vvar{y} don't have
  the same size.
\end{defun2}

\begin{defun2}{void}{app}{(`b \vvar{f}(`a),`a {?}`r \vvar{x});}
  Apply \vvar{f} to each element of \vvar{x} in turn.  Note, the
  function \vvar{f} has return type \code{`b}; this is the difference
  between \code{app} and \code{iter}.
\end{defun2}

\begin{defun2}{void}{app_c}{(`c \vvar{f}(`a,`b),`a \vvar{env},`b {?}`r \vvar{x});}
  A version of \code{app} where the function requires an environment.
\end{defun2}

\begin{defun2}{void}{iter}{(void \vvar{f}(`a),`a {?}`r \vvar{x});}
  Apply \vvar{f} to each element of \vvar{x} in turn.  Note, the
  function \vvar{f} has return type \texttt{void}; this is the
  difference between \code{iter} and \code{app}.
\end{defun2}

\begin{defun2}{void}{iter_c}{(void \vvar{f}(`b,`a),`b \vvar{env},`a {?}`r \vvar{x});}
  A version of \code{iter} where the function requires an environment.
\end{defun2}


\begin{defun2}{void}{app2}{(`c \vvar{f}(`a,`b),`a {?}`r1 \vvar{x},`b {?}`r2 \vvar{y});}
  A version of \code{app} in which the function is applied to two arrays.
\end{defun2}


\begin{defun2}{void}{app2_c}{(`d \vvar{f}(`a,`b,`c),`a \vvar{env},`b {?}`r1 \vvar{x},`c {?}`r2 \vvar{y});}
  A version of \code{app2} where the function requires an environment.
\end{defun2}


\begin{defun2}{void}{iter2}{(void \vvar{f}(`a,`b),`a {?}`r1 \vvar{x},`b {?}`r2 \vvar{y});}
  A version of \code{iter} in which the function is applied to two arrays.
\end{defun2}


\begin{defun2}{void}{iter2_c}{(void \vvar{f}(`a,`b,`c),`a \vvar{env},`b {?}`r1 \vvar{x},`c {?}`r2 \vvar{y});}
  A version of \code{iter2} where the function requires an environment.
\end{defun2}


\subsubsection*{Fold}

\begin{defun2}{`a}{fold_left}{(`a \vvar{f}(`a,`b),`a \vvar{accum},`b {?}`r \vvar{x});}
  If \vvar{x} is an array \texttt{\lb$x_1$,$x_2$,\ldots,$x_n$\rb},
  return \texttt{\vvar{f}(\ldots\vvar{f}(\vvar{f}(\vvar{a},$x_1$),$x_2$),\ldots, $x_n$)\ldots)}.
\end{defun2}

\begin{defun2}{`a}{fold_left_c}{(`a \vvar{f}(`c,`a,`b),`c \vvar{env},`a \vvar{accum},`b {?}`r \vvar{x});}
  A version of \code{fold_left} where the function \vvar{f} takes an
  environment.
\end{defun2}

\begin{defun2}{`b}{fold_right}{(`b \vvar{f}(`a,`b),`a {?}`r \vvar{x},`b \vvar{accum});}
  If \vvar{x} is an array \texttt{\lb$x_1$,$x_2$,\ldots,$x_n$\rb},
  return \texttt{\vvar{f}($x_1$, \vvar{f}($x_2$, \ldots, \vvar{f}($x_n$,\vvar{a})\ldots))}.
\end{defun2}
\begin{defun2}{`b}{fold_right_c}{(`b \vvar{f}(`c,`a,`b),`c \vvar{env},`a {?}`r \vvar{x},`b \vvar{accum});}
  A version of \code{fold_right} where the function \vvar{f} takes an
  environment.
\end{defun2}

\subsubsection*{Reversing}

\begin{defun2}{`a ?}{rev_copy}{(`a::B {?}`r \vvar{x});}
  Return a copy of an array in reverse.
\end{defun2}

\begin{defun2}{void}{imp_rev}{(`a::B {?}`r \vvar{x});}
  Imperatively reverse an array.
\end{defun2}

\subsubsection*{Predicates}

\begin{defun2}{bool}{forall}{(bool \vvar{pred}(`a),`a {?}`r \vvar{x});}
  Given a predicate on 'a values, determine whether every element in
  an array satisfies the predicate.
\end{defun2}

\begin{defun2}{bool}{forall_c}{(bool \vvar{pred}(`a,`b),`a \vvar{env},`b {?}`r \vvar{x});}
  A version of \code{forall} where the predicate takes an environment.
\end{defun2}

\begin{defun2}{bool}{memq}{(`a {?}`r \vvar{l}, `a \vvar{x});}
  Determine whether \vvar{x} is an element of array \vvar{l}, using
  physical equality (\texttt{==}) for comparison.
\end{defun2}

\begin{defun2}{bool}{mem}{(int \vvar{cmp}(`a,`a), `a {?}`r \vvar{l}, `a \vvar{x});}
  Determine whether \vvar{x} is an element of array \vvar{l},
  using the function \vvar{cmp} for comparison.
\end{defun2}

\begin{defun2}{bool}{exists}{(bool \vvar{pred}(`a),`a {?}`r \vvar{x});}
  Given a predicate on 'a values, determine whether there exists an
  element in the array that satisfies the predicate.
\end{defun2}
\begin{defun2}{bool}{exists_c}{(bool \vvar{pred}(`a,`b),`a \vvar{env},`b {?}`r \vvar{x});}
\end{defun2}

\subsubsection*{Arrays and Lists}

\begin{defun2}{`a ?}{from_list}{(List::glist_t<`a,`r> \vvar{x});}
  Build a heap-allocated array from the list \vvar{x}; the elements
  of the array are the elements of \vvar{x}.
\end{defun2}

\begin{defun2}{List::list_t<`a>}{to_list}{(`a {?}`r \vvar{x});}
  Build a heap-allocated list from the array \vvar{x}; the elements of
  the list are the elements of \vvar{x}.
\end{defun2}

\subsubsection*{Arrays of Pairs, Pairs of Arrays}

\begin{defun2}{\$(`a,`b) ?}{zip}{(`a {?}`r1 \vvar{x},`b {?}`r2 \vvar{y});}
  Given [|x1,...,xn|] and [|y1,...,yn|], return [|(x1,y1),...,(xn,yn)|].  
  Throws Array_mismatch if the lengths are not the same.
\end{defun2}

\begin{defun2}{\$(`a?,`b?)}{split}{(\$(`a,`b) {?}`r \vvar{x});}
  Given [|(x1,y1),...,(xn,yn)|], return ([|x1,...,xn|],[|y1,...,yn|]) 
\end{defun2}

\subsubsection*{Sorting}

\begin{defun2}{void}{qsort}{(int \vvar{cmp}(`a,`a), `a ?\vvar{x}, int \vvar{ofs}, int \vvar{len});}
  Does an in-place sort of a segment of array \vvar{x}, starting at the
  element with index \vvar{ofs}, and continuing for \vvar{len} elements.
  The function \vvar{cmp} should return an integer less than, equal
  to, or greater than 0 if its first argument is less than, equal to,
  or greater than its second argument, respectively.  \code{qsort}
  throws \code{Core::InvalidArg("Array::qsort")} if \vvar{len} is
  negative or \vvar{ofs} and \vvar{len} specify a segment outside the
  bounds of \vvar{x}.
\end{defun2}

\begin{defun2}{void}{msort}{(int \vvar{cmp}(`a,`a), `a ?\vvar{x}, int \vvar{ofs}, int \vvar{len});}
  Does an in-place sort of a segment of array \vvar{x}, starting at the
  element with index \vvar{ofs}, and continuing for \vvar{len} elements.
  The function \vvar{cmp} should return an integer less than, equal
  to, or greater than 0 if its first argument is less than, equal to,
  or greater than its second argument, respectively.  \code{msort}
  throws \code{Core::InvalidArg("Array::msort")} if \vvar{len} is
  negative or \vvar{ofs} and \vvar{len} specify a segment outside the
  bounds of \vvar{x}.
\end{defun2}

\subsection{Bitvec}

Namespace \code{Bitvec} implements bit vectors, which can be used to
represent sets with a fixed number of elements.

\subsubsection*{Header}
\begin{verbatim}
#include <bitvec.h>
using Bitvec;
\end{verbatim}

\subsubsection*{Types}
\begin{verbatim}
typedef int BITVEC[?];
\end{verbatim}

\subsubsection*{Creating bit vectors}

\subsubsection*{Functions}
\begin{verbatim}
BITVEC new_empty(int);
BITVEC new_full(int);
BITVEC new_copy(BITVEC);
BITVEC from_list<`a,`b>(Dict::Dict<`a,`b> d, int f(`b),int sz, List::list<`a>);
List::list<int> to_sorted_list(BITVEC bvec, int sz);
\end{verbatim}

\subsubsection*{Use}

\begin{defun}{new_empty}{(n)}
Returns a bit vector containing \var{n} bits, all set to 0.
\end{defun}

\begin{defun}{new_full}{(n)}
Returns a bit vector containing \var{n} bits, all set to 1.
\end{defun}

\begin{defun}{new_copy}{(v)}
Returns a copy of bit vector \var{v}.
\end{defun}

\subsubsection*{Accessing and updating bits}

\subsubsection*{Functions}
\begin{verbatim}
bool get(BITVEC, int);
void set(BITVEC, int);
void clear(BITVEC, int);
bool get_and_set(BITVEC, int);
void clear_all(BITVEC);
void set_all(BITVEC);
bool all_set(BITVEC bvec, int sz);
\end{verbatim}

\subsubsection*{Use}

\begin{defun}{get}{(v,n)}
Returns the \var{n}th bit of vector \var{v}.
\end{defun}

\begin{defun}{set}{(v,n)}
Sets the \var{n}th bit of vector \var{v} to 1.
\end{defun}

\begin{defun}{clear}{(v,n)}
Sets the \var{n}th bit of vector \var{v} to 0.
\end{defun}

\begin{defun}{get_and_set}{(v,n)}
Sets the \var{n}th bit of vector \var{v} to 1, and returns the value
that the \var{n}th bit had before it was set to 1.
\end{defun}

\begin{defun}{set_all}{(v)}
Sets every bit of \var{v} to 1.
\end{defun}

\begin{defun}{clear_all}{(v)}
Sets every bit of \var{v} to 0.
\end{defun}

\begin{defun}{all_set}{(v,n)}
Does ?????  FIX
\end{defun}

\subsubsection*{Combining and comparing bit vectors}

\subsubsection*{Functions}
\begin{verbatim}
void union_two(BITVEC dest, BITVEC src1, BITVEC src2);
void intersect_two(BITVEC dest, BITVEC src1, BITVEC src2);
void diff_two(BITVEC dest, BITVEC src1, BITVEC src2);
bool compare_two(BITVEC src1, BITVEC src2);
\end{verbatim}

\subsubsection*{Use}

\begin{defun}{union_two}{(dest,src1,src2)}
Updates \var{dest} with the bitwise-or of \var{src1} and \var{src2}.
There is no error if \var{dest} is shorter than \var{src1} or
\var{src2} (extra bits are ignored).  If \var{src1} or \var{src2}
is shorter than \var{dest}, \code{Null_Exception} is thrown;
\var{dest} may be partially updated.
\end{defun}

\begin{defun}{intersect_two}{(dest,src1,src2)}
Updates \var{dest} with the bitwise-and of \var{src1} and \var{src2}.
Its error conditions are the same as \code{union_two}.
\end{defun}

\begin{defun}{diff_two}{(dest,src1,src2)}
Updates \var{dest} with the bitwise-difference of \var{src1} and
\var{src2}.  Its error conditions are the same as \code{union_two}.
\end{defun}

\begin{defun}{compare_two}{(src1,src2)}
Returns 1 if every bit of
\var{src1} is the same as the corresponding bit of \var{src2}, and 0
otherwise.  If \var{src2} is longer than \var{src1}, its extra bits are
ignored.  If \var{src2} is shorter than \var{src1},
\code{Null_Exception} is thrown.
\end{defun}

\subsubsection*{Printing bit vectors}

\subsubsection*{Functions}
\begin{verbatim}
void print_bvec(BITVEC bvec);
\end{verbatim}

\subsubsection*{Use}

\begin{defun}{print_bvec}{(v)}
FIX: PUT SOMETHING HERE
\end{defun}

\subsection{Buffer}

Namespace \code{Buffer} implements extensible buffers of characters.

\subsubsection*{Header}
\begin{verbatim}
#include <buffer.h>
using Buffer;
\end{verbatim}

\subsubsection*{Types}
\begin{verbatim}
struct t;
typedef struct t @T;
\end{verbatim}

The type \code{struct t} is the abstract type of buffers.  Most
functions manipulate the type \code{T}, a non-null pointer to a buffer.

\subsubsection*{Functions}
\begin{verbatim}
T create(uint n);
string contents(T);
Core::size_t length(T);
void clear(T);
void reset(T);
void add_char(T,char);
void add_substring(T,string,int offset,int len);
void add_string(T,string);
void add_buffer(T buf_dest, T buf_source);
\end{verbatim}

\subsubsection*{Use}

\begin{defun}{create}{(n)}
Returns an empty buffer with internal storage of size \var{n}.  If
\var{n} is negative, no exception is thrown; a buffer with a small
amount of internal storage is returned instead.
\end{defun}

\begin{defun}{contents}{(b)}
Returns the contents of buffer \var{b} as a string.
\end{defun}

\begin{defun}{length}{(b)}
Returns the current length of buffer \var{b}.
\end{defun}

\begin{defun}{clear}{(b)}
Sets the length of \var{b} to 0, without modifying its internal storage.
\end{defun}

\begin{defun}{reset}{(b)}
Restores \var{b} to its initial internal storage.
\end{defun}

\begin{defun}{add_char}{(b,c)}
Adds the character \var{c} to the end of buffer \var{b}.
\end{defun}

\begin{defun}{add_substring}{(b,s,offset,len)}
Adds the substring of \var{s} starting at element \var{offset} and
continuing for \var{len} characters to the end of buffer \var{b}.  If
\var{offset} and \var{len} do not specify a valid substring of \var{s},
then the function throws \code{InvalidArg("Buffer::add_substring")}.
Note, the substring specified by \var{offset} and \var{len} may contain
null (0) characters; in any case, the entire substring is appended to
\var{b}, not just the substring up to the first null character.
\end{defun}

\begin{defun}{add_string}{(b,s)}
Adds the string \var{s} to the end of buffer \var{b}.
\end{defun}

\begin{defun}{add_buffer}{(buf_dest,buf_source)}
Adds the contents of buffer \var{buf_source} to the end of buffer
\var{buf_dest}.
\end{defun}

\subsection{Ctype}

Namespace \code{Ctype} implements the character library of C\@.

\subsubsection*{Header}
\begin{verbatim}
#include <ctype.h>
using Ctype;
\end{verbatim}

\subsubsection*{Functions}
\begin{verbatim}
int isalnum(int c));
int isalpha(int c));
int iscntrl(int c));
int isdigit(int c));
int isgraph(int c));
int islower(int c));
int isprint(int c));
int ispunct(int c));
int isspace(int c));
int isupper(int c));
int isxdigit(int c));
int tolower(int c));
int toupper(int c));
int isascii(int c));
int toascii(int c));
\end{verbatim}

\subsubsection*{Use}

\begin{defun}{isalnum}{(c)}
Returns 1 if \code{isalpha(c)} or \code{isnum(c)} returns 1,
and returns 0 otherwise.
\end{defun}

\begin{defun}{isalpha}{(c)}
Returns 1 if \code{islower(c)} or \code{isupper(c)} returns 1,
and returns 0 otherwise.
\end{defun}

\begin{defun}{iscntrl}{(c)}
Returns 1 if \var{c} is a control character,
and returns 0 otherwise.
\end{defun}

\begin{defun}{isdigit}{(c)}
Returns 1 if \var{c} is a decimal digit (\code{'0'}--\code{'9'}), and
returns 0 otherwise.
\end{defun}

\begin{defun}{isgraph}{(c)}
Returns 1 if \var{c} is a non-space printing character, and returns 0
otherwise.
\end{defun}

\begin{defun}{islower}{(c)}
Returns 1 if \var{c} is a lowercase letter (\code{'a'}--\code{'z'}),
and returns 0 otherwise.
\end{defun}

\begin{defun}{isprint}{(c)}
Returns 1 if \var{c} is a printing character (including space), and
returns 0 otherwise.
\end{defun}

\begin{defun}{ispunct}{(c)}
Returns 1 if \var{c} is a non-whitespace, non-digit, non-alpha
printing character, and returns 0 otherwise.
\end{defun}

\begin{defun}{isspace}{(c)}
Returns 1 if \var{c} is a whitespace character (\code{' '},
\verb|'\f'|, \verb|'\n'|, \verb|'\r'|, \verb|'\t'|, or \verb|'\v'|),
and returns 0 otherwise.
\end{defun}

\begin{defun}{isupper}{(\var{c})}
Returns 1 if \var{c} is an uppercase letter
(\code{'A'}--\code{'Z'}), and returns 0 otherwise.
\end{defun}

\begin{defun}{isxdigit}{(c)}
Returns 1 if \var{c} is a hexidecimal digit (\code{'0'}--\code{'9'} or
\code{'A'}--\code{'F'} or \code{'a'}--\code{'f'}), and returns 0
otherwise.
\end{defun}

\begin{defun}{tolower}{(c)}
If \var{c} is an uppercase letter (\code{'A'}--\code{'Z'}), then
\code{tolower(\var{c})} returns its lowercase (\code{'a'}--\code{'z'})
equivalent; otherwise it returns \var{c}.
\end{defun}

\begin{defun}{toupper}{(c)}
If \var{c} is a lowercase letter (\code{'a'}--\code{'z'}), then
\code{toupper(\var{c})} returns its uppercase
(\code{'A'}--\code{'Z'}) equivalent; otherwise it returns \var{c}.
\end{defun}

\subsection{Core}

The \code{Core} namespace contains fundamental types and functions.

\subsubsection*{Header}
\begin{verbatim}
#include <core.h>
using Core;
\end{verbatim}

\subsubsection*{Booleans}
\subsubsection*{Types}
\begin{verbatim}
typedef int bool;
\end{verbatim}

\subsubsection*{Values}
\begin{verbatim}
#define false (0)
#define true (1)
\end{verbatim}

\subsubsection*{Functions}
\begin{verbatim}
bool true_f<`a>(`a);
bool false_f<`a>(`a);
\end{verbatim}

\subsubsection*{Exceptions}
\begin{verbatim}
xenum exn { FileOpenError(string) };
xenum exn { FileCloseError };
\end{verbatim}

\subsubsection*{Values}
\begin{verbatim}
FILE @cyc_stdout;
FILE @cyc_stdin;
FILE @cyc_stderr;
#define EOF ((char)(-1))
\end{verbatim}

\subsubsection*{Functions}
\begin{verbatim}
int   f_string_read(FILE *,string,int,int);
int   f_string_write(FILE *,string,int,int);
int   f_seek(FILE *,int);
int   fflush(FILE *);
int   fgetc(FILE *);
FILE @file_open(string fname, string mode);
void  file_close(FILE @);
int   file_delete(string);
int   file_length(string);
\end{verbatim}

\subsubsection*{Opt}
\subsubsection*{Types}
\begin{verbatim}
struct Opt<`a> { `a v; };
typedef struct Opt<`a> *Opt_t<`a>;
\end{verbatim}

\subsubsection*{Functions}
\begin{verbatim}
Opt_t<`b> opt_map<`a,`b>(`b f(`a), Opt_t<`a> x);
\end{verbatim}

\subsubsection*{Core exceptions}
\exindex{InvalidArg}
\exindex{Failure}
\exindex{Impossible}
\exindex{Not_found}
\exindex{Unreadchable}
\begin{verbatim}
xenum exn { InvalidArg(string) };
xenum exn { Failure(string) };
xenum exn { Impossible(string) };
xenum exn { Not_found };
xenum exn { Unreachable(string) };
\end{verbatim}

\subsubsection*{Boxing and unboxing}
\subsubsection*{Functions}
\begin{verbatim}
Char      box_char(char);
char      unbox_char(Char);
Short     box_short(short);
short     unbox_short(Short);
Long Long box_long_long(long long);
long long unbox_long_long(Long Long);
Float     box_float(float);
float     unbox_float(Float);
Double    box_double(double);
double    unbox_double(Double);
\end{verbatim}

\subsubsection*{Miscellaneous types}
\begin{verbatim}
typedef unsigned int uint;
typedef unsigned int size_t;
\end{verbatim}

\subsubsection*{Miscellaneous string stuff}
\subsubsection*{Types}
\begin{verbatim}
typedef char Cstring[];
typedef char string[?];
\end{verbatim}

\subsubsection*{Functions}
\begin{verbatim}
Cstring string_to_Cstring(string);
string  new_string(int);
int     int_of_string(string);
string  string_of_int(int);
string  string_of_uint(unsigned int);
string  string_of_char(char);
\end{verbatim}

\subsubsection*{System interface}
\subsubsection*{Functions}
\begin{verbatim}
int    system(Cstring);
string get_env(string);
string std_args()[?];
\end{verbatim}

\subsubsection*{Miscellaneous functions}
\begin{verbatim}
`a   fst<`a,`b>($(`a,`b)@);
`b   snd<`a,`b>($(`a,`b)@);
`c   third<`a,`b,`c>($(`a,`b,`c)@);
`a   identity<`a>(`a);
int  intcmp(int,int);
int  charcmp(char,char);
bool is_space(char);
\end{verbatim}

\subsection{Dict}

Namespace \code{Dict} defines \emph{dictionaries}, which map keys to
values.  The dictionaries are implemented functionally: adding a mapping
to an existing dictionary produces a new dictionary, without affecting
the existing dictionary.

\subsubsection*{Header}
\begin{verbatim}
#include <dict.h>
using Dict;
\end{verbatim}

\subsubsection*{Types}
\tindex{Dict}
\tindex{struct dict}
\begin{verbatim}
struct dict<`a,`b>;
typedef struct dict<`a,`b> @Dict<`a,`b>;
\end{verbatim}

\subsubsection*{Exceptions}
\exindex{Absent}
\exindex{Present}
\begin{verbatim}
xenum exn {Absent};
xenum exn {Present};
\end{verbatim}

The exception \code{Absent} is thrown when a key is expected in a
dictionary, but is not present.  Conversely, \code{Present} is
typically thrown when a key is not expected, but is present.


\subsubsection*{Creating dictionaries}

\subsubsection*{Functions}
\begin{verbatim}
Dict<`a,`b> empty<`a,`b>(int cmp(`a,`a));
Dict<`a,`b> singleton<`a,`b>(int cmp(`a,`a),`a key,`b data);
\end{verbatim}

\subsubsection*{Use}
\begin{defun}{empty}{(cmp)}
Creates a new dictionary.  The function \var{cmp} takes two keys as
arguments, and it should return an integer less than, equal to, or
greater than 0 according to whether its first argument is less than,
equal to, or greater than the second, respectively.
\end{defun}

\begin{defun}{singleton}{(cmp,key,data)}
Creates a new dictionary with a single entry, mapping \var{key} to
\var{data}.  The function \var{cmp} should be a comparison function for
keys, as described for \code{empty} above.
\end{defun}

\subsubsection*{Accessing and updating dictionaries}

\subsubsection*{Functions}
\begin{verbatim}
bool            is_empty<`a,`b>(Dict<`a,`b> d);
bool            member<`a,`b>(Dict<`a,`b> d,`a key);
`b              lookup<`a,`b>(Dict<`a,`b> d,`a key);
Core::Opt_t<`b> lookup_opt<`a,`b>(Dict<`a,`b> d,`a key);
Dict<`a,`b>     insert<`a,`b>(Dict<`a,`b> d,`a key,`b data);
Dict<`a,`b>     insert_new<`a,`b>(Dict<`a,`b> d,`a key,`b data);
Dict<`a,`b>     inserts<`a,`b>(Dict<`a,`b> d,list<$(`a,`b)@> kds);
Dict<`a,`b>     delete<`a,`b>(Dict<`a,`b> d,`a key);
Dict<`a,`b>     delete_present<`a,`b>(Dict<`a,`b> d,`a key);
\end{verbatim}

\subsubsection*{Use}

\begin{defun}{is_empty}{(d)}
Returns 1 if \var{d} is an empty dictionary, and returns 0 if \var{d} is
not empty.
\end{defun}

\begin{defun}{member}{(d,key)}
Returns 1 if \var{key} is present in dictionary \var{d}, and returns 0
if \var{key} is not present.
\end{defun}

\begin{defun}{lookup}{(d,key)}
Looks up the value associated with \var{key} in dictionary \var{d}, and
returns it.  If \var{key} has no value in \var{d}, then \code{Absent} is
thrown.
\end{defun}

\begin{defun}{lookup_opt}{(d,key)}
Looks up the value associated with \var{key} in dictionary \var{d}, and
returns a pointer to it.  If \var{key} has no value in \var{d}, then
\code{NULL} is returned.
\end{defun}

\begin{defun}{insert}{(d,key,data)}
Adds a mapping from \var{key} to \var{data} in dictionary \var{d},
overwriting any previous value for \var{key}.
\end{defun}

\begin{defun}{insert_new}{(d,key,data)}
Behaves like \code{insert}, except that \code{Present} is thrown if
\var{key} already has a value in \var{d}.
\end{defun}

\begin{defun}{inserts}{(d,kds)}
Is used to add a list of mappings to dictionary \var{d}; \var{kds}
should be a list of (\var{key},\var{data}) pairs.
\end{defun}

\subsubsection*{Operating on dictionaries}

\subsubsection*{Functions}
\begin{verbatim}
void        app<`a,`b,`c>(`c f(`a,`b),Dict<`a,`b> d);
void        app_c<`a,`b,`c,`d>(`c f(`d,`a,`b),`d env,Dict<`a,`b> d);
void        iter<`a,`b>(void f(`a,`b),Dict<`a,`b> d);
void        iter_c<`a,`b,`c>(void f(`c,`a,`b),`c env,Dict<`a,`b> d);
Dict<`a,`c> map<`a,`b,`c>(`c f(`b),Dict<`a,`b> d);
Dict<`a,`c> map_c<`a,`b,`c,`d>(`c f(`d,`b),`d env,Dict<`a,`b> d);
`c          fold<`a,`b,`c>(`c f(`a,`b,`c),Dict<`a,`b> d,`c accum);
`c          fold_c<`a,`b,`c,`d>(`c f(`d,`a,`b,`c),`d env,
                                Dict<`a,`b> dict,`c accum);
$(`a,`b)@   choose<`a,`b>(Dict<`a,`b> d);
list<$(`a,`b)@> to_list<`a,`b>(Dict<`a,`b> d);
\end{verbatim}

\subsubsection*{Use}

\begin{defun}{app}{(f,d)}
Applies the function \var{f} to every (\var{key},\var{data}) pair in
\var{d}.  The return value of \var{f} is ignored.
\end{defun}

\begin{defun}{app_c}{(f,env,d)}
Applies the function \var{f} to \var{env} and every
(\var{key},\var{data}) pair in \var{d}.  The argument \var{env} is
intended to hold auxilliary data for \var{f}.  The return value of
\var{f} is ignored.
\end{defun}

\begin{defun}{iter}{(f,d)}
Acts just like \code{app}, except that \var{f} can return void.
(\code{app} cannot be used on functions that return void because of
restrictions in the Cyclone type system.)
\end{defun}

\begin{defun}{iter_c}{(f,env,d)}
Acts like \code{app_c} except that \var{f} can return void.
\end{defun}

\begin{defun}{map}{(f,d)}
Creates a new dictionary with a pair (\var{key},\var{f}(\var{data})) for
each pair (\var{key},\var{data}) in \var{d}.
\end{defun}

\begin{defun}{map_c}{(f,env,d)}
Does the same but applies \var{f} to \var{env} as well as \var{data}.
\end{defun}


\begin{defun}{fold}{(f,d,accum)}
Returns the value accumulated by applying \var{f} to the keys and data
in \var{d}.  For example, if \var{d} has keys \var{k1} through \var{kn}
and associated values \var{data1} through \var{datan}, then \code{fold}
returns

\begin{verbatim}
f(k1,data1,f(k2,data2, ... f(kn,datan,accum) ... ))
\end{verbatim}

\noindent The order in which the keys and data are passed to \var{f} is
not specified.
\end{defun}

\begin{defun}{fold_c}{(f,env,d,accum)}
Computes

\begin{verbatim}
f(env,k1,data1, ... f(env,kn,datan,accum) ... )
\end{verbatim}
\end{defun}

\begin{defun}{choose}{(d)}
Returns a (\var{key},\var{data}) pair from \var{d}---exactly which pair
is unspecified.  If \var{d} is empty, then \code{Absent} is thrown.
\end{defun}

\begin{defun}{to_list}{(d)}
Returns a list of the (\var{key},\var{data}) pairs of \var{d}.
\end{defun}

\subsection{Filename}

The \code{Filename} namespace contains functions for operating on
strings representing file names.

\subsubsection*{Header}
\begin{verbatim}
#include <filename.h>
using Filename;
\end{verbatim}

\subsubsection*{Functions}
\begin{alltt}
string \textbf{concat}(string, string);
string \textbf{chop_extension}(string);
string \textbf{dirname}(string);
string \textbf{basename}(string);
bool   \textbf{check_suffix}(string, string);
string \textbf{gnuify}(string);
\end{alltt}

\subsection{Fn}

Namespace \code{Fn} supports \emph{closures}, which allow a function to
be packaged up with some hidden data.

\subsubsection*{Header}
\begin{verbatim}
#include <fn.h>
using Fn;
\end{verbatim}

\subsubsection*{Types}
\begin{verbatim}
enum Function<`arg,`res> {
  Fun<`env>(`res (@)(`env,`arg), `env)
};
typedef enum Function<`arg,`res> fn<`arg,`res>;
\end{verbatim}

\subsubsection*{Functions}
\begin{verbatim}
fn<`arg,`res>    make_fn<`arg,`res,`env>(`res f(`env,`arg), `env x);
fn<`arg,`res>    fp2fn<`arg,`res>(`res f(`arg));
`res             apply<`arg,`res>(fn<`arg,`res> f, `arg x);
fn<`a,`c>        compose<`a,`b,`c>(fn<`a,`b> g, fn<`b,`c> f);
fn<`a,fn<`b,`c>> curry<`a,`b,`c>(fn<$(`a,`b)@,`c> f);
fn<$(`a,`b)@,`c> uncurry<`a,`b,`c>(fn<`a,fn<`b,`c>> f);
List::list<`b>   map_fn<`a,`b>(fn<`a,`b> f,List::list<`a> x);
\end{verbatim}

\subsubsection*{Use}
\begin{defun}{make_fn}{(f,e)}
Creates and returns a new closure for function pointer \var{f} and environment
\var{e}.
\end{defun}

\begin{defun}{fp2fn}{(f)}
Creates and returns a new closure for a function pointer that does not
require an environment.
\end{defun}

\begin{defun}{apply}{(f,x)}
Applies closure \var{f} to argument \var{x} and returns the result.  The
function pointer of \var{f} is applied to the environment of \var{f} and
the argument \var{x}.
\end{defun}

\begin{defun}{compose}{(g,f)}
Returns a new closure that composes the closures \var{f} and \var{g}:
\code{apply(compose(\var{g},\var{f}),\var{x})} should be the same as
\code{apply(\var{g},apply(\var{f},\var{x}))}.
\end{defun}

\begin{defun}{curry}{(f)}
\var{f} should be a closure that takes a pointer to a pair as an
argument.  \code{curry(\var{f})} returns a new closure that takes the
two components of a pair in two steps (applies) before passing them to
\var{f}.  That is, \code{apply(\var{f},\&\$(\var{x},\var{y}))} should be
the same as \code{apply(apply(curry(\var{f}),\var{x}),\var{y})}.
\end{defun}

\begin{defun}{uncurry}{(f)}
\code{uncurry} is the opposite of \code{curry}.  It takes a closure that
takes two arguments in sequence and returns a closure that takes the
arguments as a pair.  \code{apply(uncurry(\var{f}),\&\$(\var{x},\var{y}))}
should be the same as \code{apply(apply(\var{f},\var{x}),\var{y})}.
\end{defun}

\begin{defun}{map_fn}{(f,x)}
Applies the closure \var{f} to each element of list \var{x} in turn,
left to right, and returns a list built from the results.
\end{defun}

\subsection{Hashtable}

Namespace \code{Hashtable} implements hash tables.

\subsubsection*{Header}
\begin{verbatim}
#include <hashtable.h>
using Hashtable;
\end{verbatim}

\subsubsection*{Types}
\tindex{struct table}
\tindex{Table}
\begin{alltt}
\textbf{struct Table}<`a,`b,`e1::E,`e2::E>;
typedef struct Table<`a,`b,`e1,`e2> @\textbf{table_t}<`a,`b,`e1,`e2>;
\end{alltt}

\subsubsection*{Functions}

\begin{alltt}
table_t<`a,`b,`e1,`e2> \textbf{create}(int sz,
                              int cmp(`a,`a;`e1),
                              int hash(`a;`e2));
void                   \textbf{insert}(table_t<`a,`b,`e1,`e2> t, `a key, `b val);
`b                     \textbf{lookup}(table_t<`a,`b,`e1,`e2> t, `a key);
void                   \textbf{resize}(table_t<`a,`b,`e1,`e2> t);
void                   \textbf{remove}(table_t<`a,`b,`e1,`e2> t, `a key);
int                    \textbf{hash_string}(string_t s);
int                    \textbf{hash_stringptr}(stringptr_t s);
void                   \textbf{iter}(void f(`a,`b), table_t<`a,`b,`e1,`e2> t);
void                   \textbf{print_table_map}(table_t<`a,`b,`e1,`e2> t,
                                       void prn_key(`a), 
                                       void prn_val(`b));
\end{alltt}

\subsubsection*{Use}

\begin{defun}{create}{(sz,cmp,hash)}
Builds a new, empty hash table with initial size \var{sz}.  \var{cmp}
should be a comparison function that returns an integer less than, equal
to, or greater than 0 according to whether its first argument is less
than, equal to, or greater than its second argument.  \code{hash} should
be a hash function on the keys of the hash table.  \code{cmp} and
\code{hash} should satisfy the following property: if
\code{cmp(\var{x},\var{y})} is 0, then \code{hash(\var{x}) ==
hash(\var{y})}.
\end{defun}

\begin{defun}{insert}{(t,key,val)}
Binds \var{key} to \var{val} in hash table \var{t}.
\end{defun}

\begin{defun}{lookup}{(t,key)}
Looks up the value associated with \var{key} in hash table \var{t}.  If
\var{key} does not appear in \var{t}, \code{Not_found} is thrown.
\end{defun}


\begin{defun}{resize}{(t)}
Increases the internal storage available to hash table \var{t}.
\code{resize} is called automatically by functions like insert when the
buckets of a hash table get large, however, it can also be called by the
programmer explicitly.
\end{defun}

\begin{defun}{remove}{(t,key)}
Removes the most recent binding of \var{key} from \var{t}.  The
next-most-recent binding of \var{key} (if any) is restored.  If
\var{key} is not bound in \var{t}, \code{remove} returns silently
without changing \var{t}.
\end{defun}


\begin{defun}{hash_string}{(s)}
Returns a hash of a string \var{s}.  It is provided as a convenience for
making hash tables mapping strings to values.
\end{defun}

\begin{defun}{iter}{(f,t)}
Applies the function \var{f} to every (\var{key},\var{value}) pair in
the hash table \var{t}.
\end{defun}

\begin{defun}{print_table_map}{(t,prn_key,prn_val)}
Used for debugging.  It prints out the internal structure of \var{t}.
\var{prn_key} must be a function that prints keys, and \var{prn_val}
must be a function that prints values.
\end{defun}

\subsection{Id}

\subsubsection*{Header}
\begin{verbatim}
#include <id.h>
using Id;
\end{verbatim}

\subsection{Lexing}

\subsubsection*{Header}
\begin{verbatim}
#include <lexing.h>
using Lexing;
\end{verbatim}

\subsection{Lineno}

\subsubsection*{Header}
\begin{verbatim}
#include <lineno.h>
using Lineno;
\end{verbatim}

\subsection{List}

Namespace \code{List} defines polymorphic lists and standard operations
over them.

\subsubsection*{Header}
\begin{verbatim}
#include <list.h>
using List;
\end{verbatim}

\subsubsection*{Types}
\tindex{struct List}
\tindex{glist_t}
\tindex{list_t}
\tindex{List_t}
\begin{verbatim}
struct List<`a,`r::R>{`a hd; struct List<`a,`r> *`r tl;};
typedef struct List<`a,`r> *`r glist_t<`a,`r>;
typedef struct List<`a,`H> *list_t<`a>;
typedef struct List<`a,`H> @List_t<`a>;
\end{verbatim}

A list is built out of ``cons cells,'' which are structures with a
head ('hd') field and a tail ('tl') field.  (The term ``cons'' comes
from Lisp.)  The head of a cons cell holds the first element of the
list, and the tail points to the remainder of the list.  Most
functions use the typedef \code{list}, which is a possibly-null
pointer to a cons cell.  \code{null} serves as the empty list.

\subsubsection*{Exceptions}
\exindex{List_empty}
\exindex{List_mismatch}
\exindex{Nth}
\begin{verbatim}
xenum exn { List_empty };
xenum exn { List_mismatch };
xenum exn { Nth };
\end{verbatim}

The \code{List_empty} exception is typically thrown by a function that
expects a non-empty list, but is passed null instead.  The
\code{List_mismatch} exception is typically thrown by a function that
expects two lists of the same length, but receives lists of different
lengths instead.  The \code{Nth} exception is thrown when
\code{list_nth} doesn't have enough elements in the list.

\subsubsection*{Basic list operations}
\subsubsection*{Functions}
\begin{verbatim}
int             length(glist_t<`a,`r> x);
`a              hd(glist_t<`a,`r> x);
glist_t<`a,`r>  tl(glist_t<`a,`r> x);
list_t<`a>      copy(glist_t<`a,`r> src);
glist_t<`a,`r2> rcopy(region_t<`r2>, glist_t<`a,`r> src);
`a              nth(glist_t<`a,`r> x,int i);
\end{verbatim}

\subsubsection*{Use}

\begin{defun}{length}{(x)}
Returns the length of the list \var{x}.
\end{defun}

\begin{defun}{hd}{(x)}
Returns the first element of list \var{x}, or throws \code{List_empty}
if \var{x} is empty.
\end{defun}

\begin{defun}{tl}{(x)}
Returns the tail of \var{x}, or throws \code{List_empty} if \var{x} is
empty.  Note, \code{hd(\var{x})} is not quite equivalent to
\code{\var{x}->hd}, because the latter will throw \code{Null_Exception}
if \var{x} is empty.
\end{defun}

\begin{defun}{copy}{(src)}
Returns a fresh copy of the list \var{src} (all of the cons cells are
duplicated).
\end{defun}

\begin{defun}{nth}{(x,n)}
Returns the \var{n}th element of list \var{x}.  The elements of \var{x}
are numbered starting with 0.  If \var{n} does not correspond to an
element of \var{x}, the exception \code{Nth} is thrown.
\end{defun}

\subsubsection*{Iteration and mapping}
\subsubsection*{Functions}
\begin{verbatim}
list_t<`b>      map(`b f(`a),glist_t<`a,`r> x);
glist_t<`b,`r2> rmap(region_t<`r2>,`b f(`a),glist_t<`a,`r> x);
list_t<`b>      map_c(`b f(`c,`a),`c env,glist_t<`a,`r> x);
glist_t<`b,`r2> rmap_c(region_t<`r2>,`b f(`c,`a),`c env,
                       glist_t<`a,`r> x);
list_t<`c>      map2(`c f(`a,`b),glist_t<`a,`r1> x,glist_t<`b,`r2> y);
glist_t<`c,`r3> rmap2(region_t<`r3>,`c f(`a,`b),
                      glist_t<`a,`r1> x,glist_t<`b,`r2> y);

void app(`b f(`a),glist_t<`a,`r> x);
void app_c(`c f(`a,`b),`a,glist_t<`b,`r> x);
void app2(`c f(`a,`b),glist_t<`a,`r1> x,glist_t<`b,`r2> y);
void app2_c(`d f(`a,`b,`c),`a env,
             glist_t<`b,`r1> x,glist_t<`c,`r2> y);

void iter(void f(`a),glist_t<`a,`r> x);
void iter_c(void f(`b,`a),`b env,glist_t<`a,`r> x);
void iter2(void f(`a,`b),glist_t<`a,`r1> x,glist_t<`b,`r2> y);
void iter2_c(void f(`a,`b,`c),`a env,
             glist_t<`b,`r1> x,glist_t<`c,`r2> y);
\end{verbatim}

\subsubsection*{Use}

\begin{defun}{map}{(f,x)}
Applies the function \var{f} to each element of list \var{x}, and
returns a new list of the results.
\end{defun}

\begin{defun}{map_c}{(f,env,x)}
A version of map where \var{f} takes an environment.
\end{defun}

\begin{defun}{map2}{(f,x,y)}
Applies \var{f} to the pairs consisting of an element from list \var{x}
and its corresponding element from list \var{y}.  The results are
returned in a new list.  If \var{x} and \var{y} don't have the same
length, it throws \code{List_mismatch}.  Note, \var{f} may be applied
some number of times before \code{List_mismatch} is thrown.
\end{defun}

\begin{defun}{app}{(f,x)}
Applies \var{f} to each element of list \var{x}.  The results of the
function application are not saved.
\end{defun}

\begin{defun}{app_c}{(f,env,x)}
A version of app, except that \var{env} is passed as an additional
argument to \var{f}, along with each element of \var{x}.
\end{defun}

\begin{defun}{app2}{(f,x,y)}
Applies f to one element of x and one element of y in turn, from left to
right in the lists x and y.  If the lists \var{x} and \var{y} are not
the same length, \code{List_mismatch} is thrown.
\end{defun}

\begin{defun}{app2_c}{(f,env,x,y)}
Applies f to env, and to one element of x and one element of y in turn,
from left to right in the lists x and y.  If the lists \var{x} and
\var{y} are not the same length, \code{List_mismatch} is thrown.
\end{defun}

\begin{defun}{iter}{(f,x)}
Like \code{app(\var{f},\var{x})} except that \var{f} returns void.
\end{defun}

\begin{defun}{iter_c}{(f,env,x)}
Like \code{app_c(\var{f},\var{env},\var{x})} except that \var{f} returns
void.
\end{defun}

\begin{defun}{iter2}{(f,x,y)}
Like \code{app2(\var{f},\var{x},\var{y})} except that \var{f} returns
void.
\end{defun}

\begin{defun}{iter2_c}{(f,env,x,y)}
Like \code{app2_c(\var{f},\var{env},\var{x},\var{y})} except that \var{f}
returns void.
\end{defun}


\subsubsection*{Fold}
\subsubsection*{Functions}
\begin{verbatim}
`a fold_left(`a f(`a,`b),`a accum,glist_t<`b,`r> x);
`a fold_left_c(`a f(`c,`a,`b),`c,`a accum,glist_t<`b,`r> x);

`b fold_right(`b f(`a,`b),glist_t<`a,`r> x,`b accum);
`b fold_right_c(`b f(`c,`a,`b),`c,glist_t<`a,`r> x,`b accum);
\end{verbatim}

\subsubsection*{Use}

\begin{defun}{fold_left}{(f,accum,x)}
Returns the value accumulated by applying \var{f} to the elements of
\var{x}.  For example, if \var{x} has elements \var{v1} through
\var{vn}, then \code{fold_left} returns

\begin{verbatim}
f(f( ... f(accum,v1) ... , vn-1), vn)
\end{verbatim}

\noindent Notice that the function is first applied to the left-most (first)
element of the list.
\end{defun}

\begin{defun}{fold_left_c}{(f,env,accum,x)}
The same as \code{fold_left} except that \var{f} takes a closure.
\end{defun}

\begin{defun}{fold_right}{(f,x,accum)}
Like \code{fold_left} except that it applies \var{f} to the elements
in \var{x} from right to left, rather than left to right:

\begin{verbatim}
f(v1,f(v2, ... f(vn,accum) ... ))
\end{verbatim}
\end{defun}

\begin{defun}{fold_right_c}{(f,env,x,accum)}
The same as \code{fold_right} except that \var{f} is given a closure
argument.
\end{defun}


\subsubsection*{Reversing lists}
\subsubsection*{Functions}
\begin{verbatim}
list_t<`a>      revappend(glist_t<`a,`r> x,list_t<`a> y);
glist_t<`a,`r2> rrevappend(region_t<`r2>,glist_t<`a,`r> x,
                           glist_t<`a,`r2> y);
list_t<`a>      rev(glist_t<`a,`r> x);
glist_t<`a,`r2> rrev(region_t<`r2>,glist_t<`a,`r> x);
glist_t<`a,`r>  imp_rev(glist_t<`a,`r> x);
\end{verbatim}

\subsubsection*{Use}

\begin{defun}{revappend}{(x,y)}
Returns the list whose elements are the reverse of the elements of
\var{x}, followed by the elements of \var{y}.
\end{defun}

\begin{defun}{rev}{(x)}
Returns a new list whose elements are the elements of \var{x} in reverse
order.
\end{defun}

\begin{defun}{imp_rev}{(x)}
An imperative version of \code{rev}; instead of allocating a new
list, the elements of \var{x} are reversed in place.
\end{defun}

\subsubsection*{Appending lists}
\subsubsection*{Functions}
\begin{verbatim}
list_t<`a>      append(glist_t<`a,`r> x,list_t<`a> y);
glist_t<`a,`r2> rappend(region_t<`r2>,
                        glist_t<`a,`r> x,glist_t<`a,`r2> y);

glist_t<`a,`r>  imp_append(glist_t<`a,`r> x,glist_t<`a,`r> y);

list_t<`a>      flatten(glist_t<glist_t<`a,`r1>,`r2> x);
glist_t<`a,`r3> rflatten(region_t<`r3>,
                         glist_t<glist_t<`a,`r1>,`r2> x);
\end{verbatim}

\subsubsection*{Use}

\begin{defun}{append}{(x,y)}
Returns a list whose elements are the elements of \var{x} and \var{y}.
The cons cells making up the returned list may be shared with \var{x} or
\var{y}.
\end{defun}

\begin{defun}{imp_append}{(x,y)}
An imperative version that modifies \var{x} so that \var{y} is
appended to it.
\end{defun}

\begin{defun}{flatten}{(x)}
\var{x} should be a list of lists.  \code{flatten} returns a list whose
elements are the elements of each element of \var{x}, from left to
right.  The returned list may share some cons cells with elements of
\var{x}.
\end{defun}

\subsubsection*{Sorting lists}
\subsubsection*{Functions}
\begin{verbatim}
list_t<`a>      merge_sort(int less_eq(`a,`a), glist_t<`a,`r> x);
glist_t<`a,`r2> rmerge_sort(region_t<`r2>, int less_eq(`a,`a), 
glist_t<`a,`r>  rimp_merge_sort(int less_eq(`a,`a), glist_t<`a,`r> x);
list_t<`a>      merge(int less_eq(`a,`a),list_t<`a> a,list_t<`a> b);
glist_t<`a,`r3> rmerge(region_t<`r3>, int less_eq(`a,`a),
                       glist_t<`a,`r1> a,glist_t<`a,`r2> b);
glist_t<`a,`r>  imp_merge(int less_eq(`a,`a),
                          glist_t<`a,`r> a,glist_t<`a,`r> b);
\end{verbatim}

\subsubsection*{Use}

\begin{defun}{merge_srt}{(less_eq,x)}
Returns a list the elements of \var{x} in sorted order, according to the
ordering function \var{less_eq}.  \var{less_eq} should return an integer
less than, equal to, or greater than 0 according to whether its first
argument is less than, equal to, or greater than its second argument.
\code{merge_sort} uses an auxiliary function,
\code{merge(\var{less_eq},\var{a},\var{b})}, to merge two lists
according an ordering function.
\end{defun}

\subsubsection*{Predicates}
\subsubsection*{Functions}
\begin{verbatim}
bool forall(bool pred(`a),glist_t<`a,`r> x);
bool forall_c(bool pred(`a,`b),`a env,glist_t<`b,`r> x);
bool exists(bool pred(`a),glist_t<`a,`r> x);
bool exists_c(bool pred(`a,`b),`a env,glist_t<`b,`r> x);
bool memq(glist_t<`a,`r> l,`a x);
bool mem(int compare(`a,`a), glist_t<`a,`r> l, `a x);

Core::opt_t<`c> check_unique(int compare(`c,`c),glist_t<`c,`r> x);
\end{verbatim}

\subsubsection*{Use}

\begin{defun}{forall}{(pred,x)}
Returns true if the function \var{pred} returns true when applied to
every element of \var{x}, and returns false otherwise.
\end{defun}

\begin{defun}{forall_c}{(pred,env,x)}
A version of forall that takes a closure for the function \var{pred}.
\end{defun}

\begin{defun}{exists}{(pred,x)}
Returns true if \var{pred} returns true on any element of the list
\var{x}.
\end{defun}

\begin{defun}{memq}{(l,x)}
Returns 1 if \var{l} is an element of list \var{x} (according to
\code{==}), and returns 0 otherwise.
\end{defun}

\begin{defun}{mem}{(compare,l,x)}
A version of memq that uses a provided comparison function.
\end{defun}

\begin{defun}{check_unique}{(compare,x)}
Checks that a list of elements is unique; it assumes that the elements
of \var{x} are sorted according to the comparison function
\var{compare}.  It returns an element that is duplicated if one exists,
otherwise it returns null.
\end{defun}

\subsubsection*{Lists and tuples}
\subsubsection*{Functions}
\begin{verbatim}
list_t<$(`a,`b)@>         zip(glist_t<`a,`r1> x,glist_t<`b,`r2> y);
glist_t<$(`a,`b)@`r4,`r3> rzip(region_t<`r3> r3, region_t<`r4> r4,
                               glist_t<`a,`r1> x, glist_t<`b,`r2> y);

$(list_t<`a>,list_t<`b>)            split(glist_t<$(`a,`b)@`r1,`r2> x);
$(list_t<`a>,list_t<`b>,list_t<`c>) split3(glist_t<$(`a,`b,`c)@`r1,`r2> x);

$(glist_t<`a,`r3>,glist_t<`b,`r4>) 
rsplit(region_t<`r3> r3, region_t<`r4> r4,
           glist_t<$(`a,`b)@`r1,`r2> x);

$(glist_t<`a,`r3>,glist_t<`b,`r4>,glist_t<`c,`r5>) 
rsplit3(region_t<`r3> r3, region_t<`r4> r4, region_t<`r5> r5,
        glist_t<$(`a,`b,`c)@`r1,`r2> x);
\end{verbatim}
  
\subsubsection*{Use}

\begin{defun}{zip}{(x,y)}
\var{x} should be a list with elements \var{x1} through \var{xn}, and
\var{y} should be a list with elements \var{y1} through \var{yn}.
\code{zip} returns a list with elements \code{\&\$(\var{x1},\var{y1})}
through \code{\&\$(\var{xn},\var{yn})}.  If \var{x} and \var{y} have
different lengths, then \code{zip(\var{x},\var{y})} will throw
\code{List_mismatch}.
\end{defun}

\begin{defun}{split}{(x)}
Performs the opposite of \code{zip}; the list \var{x} is a list of
pairs, and \code{split} returns a pair of lists.
\end{defun}

\begin{defun}{split3}{(x)}
A version of split that takes a list of triples and returns a triple
of lists.
\end{defun}

  
\subsubsection*{Association lists}
\subsubsection*{Functions}
\begin{verbatim}
`b   assoc(glist_t<$(`a,`b)@`r1,`r2> l,`a x);
`b   assoc_cmp(int cmp(`a,`a), glist_t<$(`a,`b)@`r1,`r2> l,`a x);
bool mem_assoc(glist_t<$(`a,`b)@`r1,`r2> l,`a x);
\end{verbatim} %$ <-- this dollar comment fixes up font highlighting

\subsubsection*{Use}

\begin{defun}{assoc}{(l,x)}
\var{l} is a list of (key,value) pairs, and \var{x} is a key.
\code{assoc} returns the value associated with \var{x} in \var{l}, or
throws \code{Core::Not_found} if no such value exists.  It uses
\code{==} to compare \var{x} to the keys in \var{l}.
\end{defun}

\begin{defun}{assoc_cmp}{(cmp,l,x)}
A variant of assoc that uses the supplied \var{cmp} function instead
of \code{==}.
\end{defun}

\begin{defun}{mem_assoc}{(l,x)}
\var{l} is a list of (key,value) pairs, and \var{x} is a key.
\code{mem_assoc} returns true if there is a value associated with
\var{x} in \var{l}.
\end{defun}


\subsubsection*{Lists and arrays}
\subsubsection*{Functions}
\begin{verbatim}
`a ?            to_array(glist_t<`a,`r> x);
`a ?`r2         rto_array(region_t<`r2> r2, glist_t<`a,`r> x);
list_t<`a>      from_array(`a ?`r arr);
glist_t<`a,`r2> rfrom_array(region_t<`r2> r2, `a ?`r arr);
\end{verbatim}

\subsubsection*{Use}

\begin{defun}{to_array}{(x)}
Makes a new array with elements drawn from list \var{x}.
\end{defun}

\begin{defun}{from_array}{(arr)}
Makes a new list with elements drawn from array \var{arr}.
\end{defun}

\subsubsection*{Tabulation}
\subsubsection*{Functions}
\begin{verbatim}
list_t<`a>     tabulate(int n, `a f(int));
list_t<`a>     tabulate_c(int n, `a f(`b,int), `b env);
glist_t<`a,`r> rtabulate(region_t<`r> r, int n, `a f(int));
glist_t<`a,`r> rtabulate_c(region_t<`r> r,int n,`a f(`b,int),`b env);
\end{verbatim}

\subsubsection*{Use}

\begin{defun}{tabulate}{(n,f)}
Does ??? FIX
\end{defun}

\begin{defun}{tabulate_c}{(n,f,env)}
Does ??? FIX
\end{defun}

\subsubsection*{Miscellaneous}
\subsubsection*{Functions}
\begin{alltt}
int  \textbf{list_cmp}(int cmp(`a,`a), glist_t<`a,`r1> l1, glist_t<`a,`r1> l2);
bool \textbf{list_prefix}(int cmp(`a,`a), glist_t<`a,`r1> l1, glist_t<`a,`r1> l2);

list_t<`a>      \textbf{filter_c}(bool f(`b,`a), `b env, glist_t<`a,`r> l);
glist_t<`a,`r2> \textbf{rfilter_c}(region_t<`r2> r2, bool f(`b,`a), 
                          `b env, glist_t<`a,`r> l);
  
list_t<`a>      \textbf{filter}(bool f(`a), glist_t<`a,`r> l);
glist_t<`a,`r2> \textbf{rfilter}(region_t<`r2> r2,bool f(`a),glist_t<`a,`r> l);
\end{alltt}

\subsubsection*{Use}

???

\subsection{Printf}

Cyclone provides safe versions of \code{printf} and \code{fprintf} as
built-in primitives.  It also provides a built-in function,
\code{xprintf}, that allocates a string, prints to it, and returns it.
Cyclone protects against format string attacks by requiring the format
string argument of the \code{printf} functions to be a string constant,
and by doing more rigorous checks than the C version of \code{printf}.

\subsubsection*{Functions}
\begin{verbatim}
int    printf(string s, ...);
int    fprintf(FILE @f, string s, ...);
string xprintf(string s, ...);
\end{verbatim}

\subsubsection*{Use}

\begin{defun}{printf}{(s,...)}
Prints to the standard output according to the format string \var{s} and
the remaining arguments.  It returns the number of characters printed,
or a negative value if an output error or encoding error occurred.
\end{defun}

\begin{defun}{fprintf}{(f,s,...)}
Prints to the stream pointed to by \var{f} according to the format
string \var{s} and the remaining arguments.  It returns the number of
characters printed, or a negative value if an output error or encoding
error occurred.
\end{defun}

\begin{defun}{xprintf}{(s,...)}
Allocates and returns a string whose contents are given by printing
according to the format string \var{s} and the remaining arguments.
FIX: is an error possible?  What do we do then?
\end{defun}

\subsubsection*{Format strings}

Cyclone requires the format string arguments of \code{printf},
\code{fprintf}, and \code{xprintf} to be string constants, and it checks
the remaining arguments for consistency at compile time.

The format string consists of ordinary (non-\code{\%}) characters, which
are copied to the output, and \emph{conversion specifications}, which
are sequences of characters introduced by \code{\%}.  Each conversion
specification `converts' zero or more arguments to the output, as
described below.

A conversion specification has the following parts, described, in order,
by regular expressions:
\begin{itemize}
\item Zero or more \emph{flags}:
[\code{-} \code{+} \emph{space} \code{\#} \code{0}]*
\item An optional minimal \emph{width}:
(\code{*} | [\code{0}-\code{9}]+)?
\item An optional \emph{precision}:
(\code{.} (\code{*} | [\code{0}-\code{9}]+)? )?
\item An optional \emph{length modifier}:
(\code{l} | \code{h} | \code{hh} )
\item A \emph{conversion specifier}:
[\code{d} \code{i} \code{o} \code{u} \code{x} \code{X} \code{f} \code{F} \code{e} \code{E} \code{g} \code{G} \code{a} \code{A} \code{c} \code{s} \code{n} \code{\%}]
\end{itemize}

The flag \code{-} causes left justification; right justification is the
default.  The flag \code{+} is valid for numeric conversions only; it
forces a \code{+} sign to be printed for positive arguments.  The flag
\emph{space} (the space character) is valid for numeric conversions only;
it causes an extra space to be printed in front of positive arguments.
It is ignored if the \code{+} flag is also given.
The flag \code{\#} is valid for \code{o}, \code{x}, \code{X}, \code{a},
\code{A}, \code{e}, \code{E}, \code{f}, \code{F}, \code{g}, \code{G}
only; it causes an ``alternative form'' to be printed.  For \code{o} it
forces a leading zero to be printed; for \code{x} and \code{X}, a
leading \code{0x} or \code{0X} is added to non-zero conversions; for the
other conversions, a decimal point is always printed.

The width of a conversion specification gives a minimum width for the
result of the conversion; if the conversion naturally gives a shorter
result, it is padded by spaces to the minimum width.  If the width has
the form \code{*}, the minimum width is given by an integer argument.

A precision is valid for numeric conversions only.  It gives the minimum
number of digits to appear for the \code{d}, \code{i}, \code{o},
\code{u}, \code{x}, and \code{X} conversions, the number of digits to
appear after the decimal point in the \code{a}, \code{A}, \code{e},
\code{E}, \code{f}, and \code{F} conversions, and the maximum number of
significant digits for the \code{g} and \code{G} conversions.  If the
precision has the form \code{.} \code{*}, it is given by an integer
argument.

The optional length modifier and conversion specifier give the type of
the corresponding argument, as shown in the table below.  If no type
appears, then the combination of conversion specifier and length
modifier is not allowed.

\begin{tt}
\begin{tabular}{r|llll}
  & none         & l             & h              & hh\\\hline
d & int          & long          & short          & char \\
i & int          & long          & short          & char \\
o & unsigned int & unsigned long & unsigned short & unsigned char \\
u & unsigned int & unsigned long & unsigned short & unsigned char \\
x & unsigned int & unsigned long & unsigned short & unsigned char \\
X & unsigned int & unsigned long & unsigned short & unsigned char \\
f & double \\
F & double \\
e & double \\
E & double \\
g & double \\
G & double \\
a & double \\
A & double \\
c & int \\
s & string \\
n & int @        & unsigned long @ & short @      & char @ \\
\% & \textrm{(no arg)}
\end{tabular}
\end{tt}

No flags, width, precision, or length modifiers are allowed for the
\code{s} and \code{\%} conversions.

The \code{d} and \code{i} specifiers print their arguments as signed
decimal integers.

The \code{o} specifier prints its argument as an unsigned octal number.

The \code{u} specifier prints its argument as an unsigned decimal number.

The \code{x} and \code{X} specifiers print their arguments as unsigned
hexidecimal integers.

The \code{f} and \code{F} specifiers print their arguments in the form
[-]ddd.ddd (no exponents).

The \code{e} and \code{E} specifiers print their arguments using exponents.

The \code{g} and \code{G} specifiers print their arguments using either
the \code{f}/\code{F}
style or the \code{e}/\code{E} style, depending on the precision and the
argument.

The \code{a} and \code{A} specifiers print their arguments as
hexidecimal floating point numbers.

The \code{c} specifier prints its argument as an unsigned character.

The \code{s} specifier prints its argument as a zero terminated string.

The \code{n} specifier does not print its argument; instead, the number
of characters written up to the \code{n} specifier is stored where the
argument points.

The \code{\%} specifier prints a \code{\%} character.


\subsection{Position}

\subsubsection*{Header}
\begin{verbatim}
#include <position.h>
using Position;
\end{verbatim}

\subsection{PP}

\subsubsection*{Header}
\begin{verbatim}
#include <pp.h>
using PP;
\end{verbatim}

\subsection{Queue}

Namespace \code{Queue} defines generic, imperative, first-in-first-out
queues.

\subsubsection*{Header}
\begin{verbatim}
#include <queue.h>
using Queue;
\end{verbatim}

\subsubsection*{Types}
\tindex{struct queue}
\tindex{Queue}
\begin{verbatim}
struct queue<`a>;
typedef struct queue<`a> @Queue<`a>;
\end{verbatim}

\subsubsection*{Exceptions}
\exindex{Empty}
\begin{verbatim}
xenum exn { Empty };
\end{verbatim}

\code{Empty} is thrown by operations that expect a non-empty queue, but
receive an empty queue.

\subsubsection*{Basic operations on queues}
\subsubsection*{Functions}
\begin{verbatim}
Queue<`a> create<`a>();
void      add<`a>(Queue<`a>,`a x);
`a        take<`a>(Queue<`a>);
`a        peek<`a>(Queue<`a>);
bool      is_empty<`a>(Queue<`a>);
void      clear<`a>(Queue<`a>);
int       length<`a>(Queue<`a>);  
\end{verbatim}

\subsubsection*{Use}

\begin{defun}{create}{()}
Creates a new, empty queue.
\end{defun}

\begin{defun}{add}{(q,x)}
Adds \var{x} to the end of queue \var{q}, by side effect.
\end{defun}

\begin{defun}{take}{(q)}
Removes an element from the front of queue \var{q} by side effect, and
returns the element.  If \var{q} is empty, \code{Empty} is thrown.
\code{peek(\var{q})} returns the element at the front of queue \var{q};
if \var{q} is empty, \code{Empty} is thrown.
\end{defun}

\begin{defun}{is_empty}{(q)}
Returns \code{true} if \var{q} is an empty queue.
\end{defun}

\begin{defun}{clear}{(q)}
Removes all elements from \var{q} by side effect.
\end{defun}

\begin{defun}{length}{(q)}
Returns the number of elements in queue \var{q}.
\end{defun}

\subsubsection*{Iterating over queues}
\subsubsection*{Functions}
\begin{verbatim}
void iter<`a>(void f(`a), Queue<`a>);
void app<`a,`b>(`b f(`a), Queue<`a>);
\end{verbatim}

\subsubsection*{Use}

\begin{defun}{iter}{(f,q)}
Applies the function \var{f} to the elements of queue \var{q}, in order
from front to back.  \var{f} is a function that returns void.
\end{defun}

\begin{defun}{app}{(f,q)}
Applies the function \var{f} to the elements of queue \var{q}, in order
from front to back.  \var{f} is a function that returns a boxed type.
\end{defun}

\subsection{Ref}

Namespace \code{Ref} defines generic reference cells.

\subsubsection*{Header}
\begin{verbatim}
#include <ref.h>
using Ref;
\end{verbatim}

\subsubsection*{Types}
\tindex{struct Reference}
\tindex{ref}
\begin{verbatim}
struct Reference<`a> {`a v;};
typedef struct Reference<`a> @ref<`a>;
\end{verbatim}

\subsubsection*{Functions}
\begin{verbatim}
ref<`a> new_ref<`a>(`a);
`a get<`a>(ref<`a>);
void set<`a>(ref<`a>,`a);
\end{verbatim}

\subsubsection*{Use}

\begin{defun}{new_ref}{(x)}
Returns a new reference cell holding \var{x}.
\end{defun}

\begin{defun}{get}{(r)}
Returns the value held in reference cell \var{r}.
\end{defun}

\begin{defun}{set}{(r,x)}
Updates the value held in cell \var{r} with \var{x}.
\end{defun}

\subsection{Rope}

Namespace \code{Rope} defines a representation of character strings that
can be concatenated in constant time.

\subsubsection*{Header}
\begin{verbatim}
#include <rope.h>
using Rope;
\end{verbatim}

\subsubsection*{Types}
\tindex{struct Rope_node}
\tindex{rope}
\begin{verbatim}
struct Rope_node;
typedef struct Rope_node @rope;
\end{verbatim}


\subsubsection*{Basic operations on ropes}
\subsubsection*{Functions}
\begin{verbatim}
uint length(rope);
int cmp(rope,rope);
\end{verbatim}

\subsubsection*{Use}

\begin{defun}{length}{(r)}
Returns the number of characters in the string represented by rope
\var{r}.
\end{defun}

\begin{defun}{cmp}{(r1,r2)}
Returns an integer less than, equal to, or greater than 0, according to
whether the string represented by \var{r1} is lexicographically less
than, equal to, or greater than the string represented by \var{r2}.
\end{defun}

\subsubsection*{Converting between ropes and strings}
\subsubsection*{Functions}
\begin{verbatim}
rope from_string(string);
string to_string(rope);
\end{verbatim}

\subsubsection*{Use}

\begin{defun}{from_string}{(s)}
Allocates and returns a new rope containing string \var{s}.
\end{defun}

\begin{defun}{to_string}{(r)}
Converts rope \var{r} to a string and returns it.
\end{defun}

\subsubsection*{Concatenation}
\subsubsection*{Functions}
\begin{verbatim}
rope concat(rope,rope);
rope concata(rope[?]);
rope concatl(List::list<rope>);
\end{verbatim}

\subsubsection*{Use}

\begin{defun}{concat}{(r1,r2)}
Returns a rope containing the concatenation of \var{r1} and \var{r2}.
\end{defun}

\begin{defun}{concata}{(arr)}
Returns a rope containing the concatenation of the ropes in array
\var{arr}, from left to right.
\end{defun}

\begin{defun}{concatl}{(l)}
Returns a rope containing the concatenation of the ropes in list
\var{l}, from left to right.
\end{defun}

\subsection{Scanf}

Cyclone provides safe versions of \code{scanf} and \code{fscanf} as
built-in primitives.  \code{sscanf} is not yet supported.

\subsubsection*{Functions}
\begin{verbatim}
int    scanf(string s,...);
int    fscanf(FILE @ f, string s,...);
\end{verbatim}

\subsubsection*{Use}

\begin{defun}{scanf}{(s,...)}
Reads from the standard input according to the format string \var{s}.
Results are stored in the remaining arguments.
\code{scanf} returns the number of arguments successfully read.
If an error occurs before any arguments are converted, \code{EOF} is
returned.
\end{defun}

\begin{defun}{fprintf}{(f,s,...)}
Reads from the stream \var{f} according to the format string \var{s}.
Results are stored in the remaining arguments.
\code{fscanf} returns the number of arguments successfully read.
If an error occurs before any arguments are converted, \code{EOF} is
returned.
\end{defun}

\subsubsection*{Format strings}

Cyclone requires the format string arguments of \code{scanf} and
\code{fscanf} to be string constants, and it checks the remaining
arguments for consistency at compile time.

The format string consists of white space characters (which match any
(possible empty) sequence of white space characters), ordinary
(non-\code{\%}) characters (which match themselves), and \emph{conversion
specifications}, which are sequences of characters introduced by
\code{\%}.  Each conversion specification `converts' a sequence of
characters from the input stream, placing the result into storage
pointed to by the corresponding argument of \code{scanf} or
\code{fscanf}.

A conversion specification has the following parts, described, in order,
by regular expressions:
\begin{itemize}
\item An optional assignment suppression character, \code{*}
\item An optional maximal \emph{width}:
([\code{0}-\code{9}]+)?
\item An optional \emph{length modifier}:
(\code{l} | \code{h} | \code{hh} )
\item A \emph{conversion specifier}:
[\code{d} \code{i} \code{o} \code{u} \code{x} \code{X} \code{f} \code{F} \code{e} \code{E} \code{g} \code{G} \code{a} \code{A} \code{n} \code{\%}]
\end{itemize}

If the assignment suppression character \code{*} is given, the
characters corresponding to the conversion specifier are read, but no
argument is assigned, and no argument is required for that specifier.

The optional length modifier and conversion specifier give the type of
the corresponding argument, as shown in the table below.  If no type
appears, then the combination of conversion specifier and length
modifier is not allowed.

\begin{tt}
\begin{tabular}{r|llll}
  & none           & l               & h                &  hh\\\hline
d & int @          & long @          & short @          & char @ \\
i & int @          & long @          & short @          & char @ \\
o & unsigned int @ & unsigned long @ & unsigned short @ & unsigned char @ \\
u & unsigned int @ & unsigned long @ & unsigned short @ & unsigned char @ \\
x & unsigned int @ & unsigned long @ & unsigned short @ & unsigned char @ \\
X & unsigned int @ & unsigned long @ & unsigned short @ & unsigned char @ \\
f & float @        & double @ \\
F & float @        & double @ \\
e & float @        & double @ \\
E & float @        & double @ \\
g & float @        & double @ \\
G & float @        & double @ \\
a & float @        & double @ \\
A & float @        & double @ \\
n & int @          & unsigned long @ & short @          & char @ \\
\% & \texttt{(no arg)}
\end{tabular}
\end{tt}

The \code{d} and \code{i} specifiers match optionally signed decimal
integers.

The \code{o} specifier matches an optionally signed octal integer.

The \code{u} specifier matches an optionally signed decimal integer.

The \code{x} and \code{X} specifiers match optionally signed hexidecimal
integers.

The \code{f}, \code{F}, \code{e}, \code{E}, \code{g}, \code{G},
\code{a}, and \code{A} specifiers match floating point numbers.

The \code{n} specifier does not match any input; instead, the number of
characters read by the call of \code{fscanf} or \code{scanf} up to the
\code{n} specifier is stored where the argument points.

The \code{\%} specifier matches a \code{\%} character.  No assignment
suppression, width, or length modifiers are allowed for the \code{\%}
conversion.

\subsection{Set}

Namespace \code{Set} defines polymorphic, functional, finite sets of
elements with a total order.

\subsubsection*{Header}
\begin{verbatim}
#include <set.h>
using Set;
\end{verbatim}

\subsubsection*{Types}
\tindex{struct set}
\tindex{Set}
\begin{verbatim}
struct set<`a>;
typedef struct set<`a> @Set<`a>;
\end{verbatim}

\subsubsection*{Exceptions}
\exindex{Absent}
\begin{verbatim}
xenum exn {Absent};
\end{verbatim}

\subsubsection*{Creating sets}

\subsubsection*{Functions}
\begin{verbatim}
Set<`a> empty<`a>(int cmp(`a,`a));
Set<`a> singleton<`a>(int cmp(`a,`a),`a x);
Set<`a> from_list<`a>(int cmp(`a,`a),list<`a> x);
\end{verbatim}

\subsubsection*{Use}

Each set creation function requires an comparison function \var{cmp}
which should return an integer less than, equal to, or greater than 0,
according to whether its first argument is less than, equal to, or
greater than its second argument.

\begin{defun}{empty}{(cmp)}
Allocates and returns a new empty set.
\end{defun}

\begin{defun}{singleton}{(cmp,x)}
Allocates and returns a new set containing the single element \var{x}.
\end{defun}

\begin{defun}{from_list}{(cmp,l)}
Allocates and returns a new set with the same elements as the list
\var{l}.
\end{defun}

\subsubsection*{Basic set operations}
\subsubsection*{Functions}
\begin{verbatim}
Set<`a> insert<`a>(Set<`a> s,`a elt);
Set<`a> delete<`a>(Set<`a> s,`a elt);
Set<`a> union_two<`a>(Set<`a> s1,Set<`a> s2);
Set<`a> intersect<`a>(Set<`a> s1,Set<`a> s2);
Set<`a> diff<`a>(Set<`a> s1,Set<`a> s2);
\end{verbatim}

\subsubsection*{Use}

\begin{defun}{insert}{(s,elt)}
Returns a set containing the elements of \var{s} plus the element
\var{elt}.  The set \var{s} is not affected.
\end{defun}

\begin{defun}{delete}{(s,elt)}
Returns a set containing all the elements of \var{s}, except for
\var{elt}.  If \var{s} does not contain \var{elt}, a set with the same
elements as \var{s} is returned.
\end{defun}

\begin{defun}{union_two}{(s1,s2)}
Returns a set whose elements are the union of the elements of \var{s1}
and \var{s2}.  The name \code{union_two} is used because \code{union} is
a keyword in Cyclone.
\end{defun}

\begin{defun}{difference}{(s1,s2)}
Returns a set whose elements are the elements of \var{s1}, except for
those elements that also in set \var{s2}.
\end{defun}

\begin{defun}{intersection}{(s1,s2)}
Returns the intersection of \var{s1} and \var{s2}.
\end{defun}

\subsubsection*{Predicates on sets}
\subsubsection*{Functions}
\begin{verbatim}
int  cardinality<`a>(Set<`a> s);
bool is_empty<`a>(Set<`a> s);
bool member<`a>(Set<`a> s,`a elt);
bool subset<`a>(Set<`a> s1,Set<`a> s2);
int  compare<`a>(Set<`a> s1,Set<`a> s2);
bool equals <`a>(Set<`a> s1,Set<`a> s2);
\end{verbatim}

\subsubsection*{Use}

\begin{defun}{cardinality}{(s)}
Returns the number of elements in set \var{s}.
\end{defun}

\begin{defun}{is_empty}{(s)}
Returns true if the set \var{s} is empty, else it returns false.
\end{defun}

\begin{defun}{member}{(s,elt)}
Returns true if \var{elt} is an element of \var{s}.
\end{defun}

\begin{defun}{subset}{(s1,s2)}
Returns true if every element of \var{s1} is an element of \var{s2}.
\end{defun}

\begin{defun}{compare}{(s1,s2)}
The function \code{compare} defines a total ordering on sets;
\code{compare(\var{s1},\var{s2})} returns an integer less than, equal
to, or greater than 0, according to whether its first argument is less
than, equal to, or greater than its second argument.
\end{defun}

\begin{defun}{equals}{(s1,s2)}
Returns true if \var{s1} is a subset of \var{s2} and \var{s2} is a
subset of \var{s1}.
\end{defun}

\subsubsection*{Iterating over sets}
\subsubsection*{Functions}
\begin{verbatim}
void app<`a,`b>(`b f(`a),Set<`a> s);
void iter<`a>(void f(`a),Set<`a> s);
void iter_c<`a,`c>(void f(`c,`a),`c env,Set<`a> s);

`b   fold<`a,`b>(`b f(`a,`b),Set<`a> s,`b accum);
`b   fold_c<`a,`b,`c>(`b f(`c,`a,`b),`c env,Set<`a> s,`b accum);
\end{verbatim}

\subsubsection*{Use}

\begin{defun}{app}{(f,s)}
Applies function \var{f} to each element of set \var{s}, in some
unspecified order.  \var{f} is a function returning a boxed value.
\end{defun}

\begin{defun}{iter}{(f,s)}
Applies function \var{f} to each element of set \var{s}, in some
unspecified order.  \var{f} is a function returning void.
\end{defun}

\begin{defun}{iter_c}{(f,env,s)}
A version of \code{iter} where \var{f} takes a closure, \var{env}.
\end{defun}

\begin{defun}{fold}{(f,s,accum)}
If the set \var{s} has elements \var{s1} through \var{sn}, then
\code{fold(\var{f},\var{s},\var{accum})} returns
\begin{verbatim}
f(s1,f(s2, ... f(sn,accum) ... ))
\end{verbatim}
\end{defun}

\begin{defun}{fold_c}{(f,env,s,accum)}
A version of \var{fold} where \var{f} takes a closure \var{env}.
\end{defun}

\subsubsection*{Extracting elements of sets}
\subsubsection*{Functions}
\begin{verbatim}
`a choose<`a>(Set<`a> s);
list<`a> elements<`a>(Set<`a> s);
\end{verbatim}

\subsubsection*{Use}

\begin{defun}{choose}{(s)}
Returns an unspecified element of set \var{s}.
\end{defun}

\begin{defun}{elements}{(s)}
Returns all of the elements of set \var{s} in some unspecified order.
\end{defun}

\subsection{Splay}

\subsubsection*{Header}
\begin{verbatim}
#include <splay.h>
using Splay;
\end{verbatim}

\subsection{String}

Namespace \code{String} implements a string library similar to the one
provided by C\@.  In Cyclone, a string is a \code{char[?]}.  Most
functions in String consider a zero (null) character as an end of string
marker, so we may have \code{strlen(\var{s}) < size(\var{s})} for a
string \var{s}.  Functions that do not follow this convention have names
starting with 'z'.  Many functions accept a string and an offset; these
functions may have names ending in 'o'.  Functions whose arguments have
range errors throw the \code{Core::InvalidArg} exception.

\subsubsection*{Header}
\begin{verbatim}
#include <string.h>
using String;
\end{verbatim}


\subsubsection*{String length}
\subsubsection*{Functions}
\begin{verbatim}
size_t strlen(string s);
size_t strleno(string s, int ofs);
\end{verbatim}

\subsubsection*{Use}

\begin{defun}{strlen}{(s)}
Returns the length of the string \var{s}.  It considers a null
character to mark the end of the string.
\end{defun}

\begin{defun}{strleno}{(s,ofs)}
Returns the length of the string starting from offset \var{ofs}.  If
\var{ofs} is out of range, \code{strleno} throws
\code{Core::InvalidArg("String::strleno")}.
\end{defun}

\subsubsection*{Comparing strings}
\subsubsection*{Functions}
\begin{verbatim}
int strcmp(string s1, string s2);
int strcmpo(string s1, int ofs1, string s2, int ofs2);
int strncmp(string s1, int ofs1, string s2, int ofs2, size_t len);
int strncasecmp(string s1, int ofs1, string s2, int ofs2, size_t len);
int zstrcmp(string,string);
int zstrncmp(string s1,string s2,size_t n);
\end{verbatim}

\subsubsection*{Use}

All of these functions return an integer less than, equal to, or greater
than 0 if their first argument is less than, equal to, or greater than
their second argument, respectively.  The ordering used is the standard,
lexicographic ordering.

\begin{defun}{strcmp}{(s1,s2)}
Compares \var{s1} to \var{s2}.
\end{defun}

\begin{defun}{strcmpo}{(s1,ofs1,s2,ofs2)}
Compares \var{s1} starting at offset \var{ofs1} to \var{s2} starting at
\var{ofs2}.  If \var{ofs1} or \var{ofs2} is out of range, \code{strcmpo}
throws \code{Core::InvalidArg("String::strcmpo")}.
\end{defun}

\begin{defun}{strncmp}{(s1,ofs1,s2,ofs2,len)}
Compares at most \var{len} characters of \var{s1} starting at offset
\var{ofs1} to \var{s2} starting at \var{ofs2}.  If \var{ofs1} or
\var{ofs2} is out of range, \code{strncmp} throws
\code{Core::InvalidArg("String::strncmp")}.  If \var{len} is negative,
\code{strncmp} returns 0.
\end{defun}

\begin{defun}{strncasecmp}{(s1,ofs1,s2,ofs2,len)}
A case-insensitive version of \code{strncmp}.
\end{defun}

\begin{defun}{zstrcmp}{(s1,s2)}
Compares \var{s1} to \var{s2}, and it assumes that null (zero)
characters in \var{s1} and \var{s2} are not end-of-string markers.
\end{defun}

\begin{defun}{zstrcmp}{(s1,s2,n)}
Compares at most \var{n} characters of \var{s1} to \var{s2}, assuming
that null characters do not mark the end of strings.  If \var{n} is less
than 0, \code{zstrcmp(\var{s1},\var{s2},\var{n})} returns 0.
\end{defun}

\subsubsection*{Concatenating strings}
\subsubsection*{Functions}
\begin{verbatim}
string strcat(string dest,string src);
string strcato(string dest,string src,int srcOfs);
string strconcat(string,string);
string strconcat_l(list<string>);
string str_sepstr(list<string>,string);
\end{verbatim}

\subsubsection*{Use}

\begin{defun}{strcat}{(dest,src)}
Concatenates \var{src} onto \var{dest} and returns \var{dest}.  If
\var{dest} is not large enough, \code{strcat} throws
\code{Core::InvalidArg("String::strcat")}.
\end{defun}

\begin{defun}{strcato}{(dest,src,srcOfs)}
Concatenates \var{src} staring at \var{srcOfs} onto \var{dest} and
returns \var{dest}.  If \var{dest} is not large enough, \code{strcat}
throws \code{Core::InvalidArg("String::strcato")}.
\end{defun}

\begin{defun}{strconcat}{(s1,s2)}
Allocates and returns a new string whose contents are the concatenation
of \var{s1} and \var{s2}.
\end{defun}

\begin{defun}{strconcat_l}{(l)}
Allocates and returns a new string whose contents are the concatenation
of the strings in the list \var{l}, from left to right.
\end{defun}

\begin{defun}{str_sepstr}{(l,s)}
Allocates and returns a new string whose contents are the concatenation
of the strings in the list \var{l}, with \var{s} used as a separator
between each two adjacent elements in \var{l}.
\end{defun}

\subsubsection*{Copying strings and substrings}
\subsubsection*{Functions}
\begin{verbatim}
string strcpy(string dest,string src); 
string strncpy(string,string,size_t);
string zstrncpy(string,string,size_t);
string expand(string s, size_t sz);
string realloc_str(string str, size_t sz);
string strdup(string src);
string substring(string,int ofs, size_t n);
\end{verbatim}

\subsubsection*{Use}

\begin{defun}{strcpy}{(dest,src)}
Copies \var{src} into \var{dest} and returns \var{dest}.  If \var{dest}
is not big enough to hold \var{src}, \code{strcpy} throws
\code{Core::InvalidArg("String::strncpy")}.
\end{defun}

\begin{defun}{strncpy}{(dest,src,len)}
Copies at most \var{len} characters of \var{src} into \var{dest} and
returns \var{dest}.  If \var{dest} is not big enough to hold
\var{src}, \code{strcpy} throws
\code{Core::InvalidArg("String::strncpy")}.
\end{defun}

\begin{defun}{zstrncpy}{(dest,src,len)}
A variant of \code{strncpy} that does not consider null characters to
terminate strings.
\end{defun}

\begin{defun}{expand}{(s,n)}
Allocates and returns a new string that has size \var{n} or
\code{strlen(\var{s})}, whichever is greater.  The returned string has
the same contents as \var{s} (considering null characters as string
terminators).
\end{defun}

\begin{defun}{realloc_str}{(s,n)}
Like expand except that the returned string is usually a bit bigger than
\var{n} and \code{strlen(\var{s})}.
\end{defun}

\begin{defun}{strdup}{(s)}
Allocates and returns a new string with the same contents as \var{s}
(assuming that null characters terminate strings).
\end{defun}

\begin{defun}{substring}{(s,ofs,len)}
Allocates and returns a new string whose contents are \var{len}
characters of \var{s} starting at offset \var{ofs}.  If \var{ofs} or
\var{len} are out of bounds then substrings throws
\code{Core::InvalidArg("String::substring")}.
\end{defun}

\subsubsection*{Transforming strings}
\subsubsection*{Functions}
\begin{verbatim}
string replace_suffix(string,string,string);
\end{verbatim}

\subsubsection*{Use}

\begin{defun}{replace_suffix}{(s,before,after)}
Allocates and returns a string whose characters are the characters of
\var{s}, with its suffix before replaced by the string after.  If before
is not a suffix of \var{s}, then \code{replace_suffix} throws
\code{Core::InvalidArg("String::replace_suffix")}.
\end{defun}

\subsubsection*{Searching in strings}
\subsubsection*{Functions}
\begin{verbatim}
int strchr(string s, int ofs, char c);
int strrchr(string s, int ofs, char c);
int strpbrk(string s, int ofs, string accept);
int strspn(string s, int ofs, string accept);
\end{verbatim}

\subsubsection*{Use}

\begin{defun}{strchr}{(s,ofs,c)}
Returns the lowest index \code{\var{i} >= \var{ofs}} such that
\code{\var{s}[\var{i}] == \var{c}}.  If \var{ofs} is out of range strchr
throws \code{Core::InvalidArg("String::strchr")}.  If \var{c} does not
appear in \var{s} starting at \var{ofs}, \code{strchr} returns -1.
\end{defun}

\begin{defun}{strrchr}{(s,ofs,c)}
Returns the greatest index \code{\var{i} >= \var{ofs}} such that
\code{\var{s}[\var{i}] == \var{c}}.  If \var{ofs} is out of range,
\code{strrchr} throws \code{Core::InvalidArg("String::strrchr")}.  If
\var{c} does not appear in \var{s} starting at \var{ofs}, \code{strrchr}
returns -1.
\end{defun}

\begin{defun}{strpbrk}{(s,ofs,accept)}
Returns the lowest index \code{\var{i} >= \var{ofs}} such that
\code{\var{s}[\var{i}] == \var{c}}, where \var{c} is any character of
accept.  If \var{ofs} is out of range, strpbrk throws
\code{Core::InvalidArg("String::strpbrk")}.  If no character of accept
appears in \var{s} starting at \var{ofs}, \code{strpbrk} returns -1.
\end{defun}

\begin{defun}{strspn}{(s,ofs,accept)}
Returns the lowest index \code{\var{i} >= \var{ofs}} such that
\code{\var{s}[\var{i}] == \var{c}}, where \var{c} is any character not
in the string \var{accept}.  If \var{ofs} is out of range, \code{strspn}
throws \code{Core::InvalidArg("String::strpbrk")}.  If no character of
\var{accept} appears in \var{s} starting at \var{ofs}, \code{strspn}
returns the length of \var{s} after \var{ofs}.
\end{defun}


\subsubsection*{String conversions}
\subsubsection*{Functions}
\begin{verbatim}
list<Char> explode(string s);
string implode(list<Char> c);
\end{verbatim}

\subsubsection*{Use}

\begin{defun}{explode}{(s)}
Returns a list of the characters in string \var{s}.
\end{defun}

\begin{defun}{implode}{(l)}
Allocates and returns a string whose contents are the characters in list
\var{l}, from left to right.
\end{defun}

\subsection{Xarray}

Namespace \code{Xarray} defines arrays that can grow: they can be
extended by new elements.  (Currently, an xarray cannot shrink.)

\subsubsection*{Header}
\begin{verbatim}
#include <xarray.h>
using Xarray;
\end{verbatim}

\subsubsection*{Types}
\tindex{struct xarray}
\tindex{Xarray}
\begin{verbatim}
struct xarray<`a> {
  `a elmts[?];
  int num_elmts;
};
typedef struct xarray<`a> @Xarray<`a>;
\end{verbatim}

\subsubsection*{Creating xarrays}
\subsubsection*{Functions}
\begin{verbatim}
Xarray<`a> create<`a>(int, `a);
Xarray<`a> create_empty<`a>();
Xarray<`a> singleton<`a>(int, `a);
\end{verbatim}

\subsubsection*{Use}

\begin{defun}{create_empty}{()}
Returns a new xarray containing no elements.
\end{defun}

\begin{defun}{create}{(n,x)}
Returns a new xarray containing no elements.  The xarray has enough
initial internal storage to hold up to \var{n} elements without
resizing; \var{x} is used internally in allocating this array.
\end{defun}

\begin{defun}{singleton}{(n,x)}
Returns a new xarray with a single element, \var{x}.  The xarray has
enough initial internal storage to hold up to \var{n} elements without
resizing.
\end{defun}

\subsubsection*{Accessing and updating xarrays}
\subsubsection*{Functions}
\begin{verbatim}
`a         get<`a>(Xarray<`a>, int);
void       set<`a>(Xarray<`a>, int, `a);
void       add<`a>(Xarray<`a>, `a);
int        add_ind<`a>(Xarray<`a>, `a);
Xarray<`a> append<`a>(Xarray<`a>, Xarray<`a>);
void       imp_append<`a>(Xarray<`a>, Xarray<`a>);
void       delete<`a>(Xarray<`a> xarr, int num);
void       remove<`a>(Xarray<`a> xarr, int i);
\end{verbatim}

\subsubsection*{Use}

\begin{defun}{get}{(a,n)}
Returns the \var{n}th element of array \var{a}.  If \var{a} does not
have an \var{n}th element, get throws InvalidArg("Xarray::get: bad
index").
\end{defun}

\begin{defun}{set}{(a,n,x)}
Updates the \var{n}th element of \var{a} to be \var{x}.  If \var{a} does
not have an \var{n}th element, set throws InvalidArg("Xarray::set: bad
index").
\end{defun}

\begin{defun}{add}{(a,x)}
Extends the array \var{a} by adding \var{x} as a new last element.
\end{defun}

\begin{defun}{add_ind}{(a,x)}
Extends the array \var{a} by adding \var{x} as a new last element.  It
returns the index of the new element.
\end{defun}

\begin{defun}{append}{(a1,a2)}
Returns a new xarray whose elements are the elements of \var{a1}
followed by the elements of \var{a2}.
\end{defun}

\begin{defun}{imp_append}{(a1,a2)}
An imperative version of append.  It extends array \var{a1} by the
elements of \var{a2}.
\end{defun}

\begin{defun}{delete}{(a,n)}
Deletes the last \var{n} elements of array \var{a}.  FIX: IT'S IN THE .H
FILE BUT NOT DEFINED IN THE .CYC FILE!  It should throw some exception
if \var{a} has less than \var{n} elements.
\end{defun}


\begin{defun}{remove}{(a,n)}
Deletes the \var{n}th element of array \var{a}; all higher elements move
down one position.  FIX: IT'S IN THE .H FILE BUT NOT DEFINED IN THE .CYC
FILE!  It should throw some exception if \var{a} has less than \var{n}
elements.
\end{defun}

\subsubsection*{Conversions on xarrays}
\subsubsection*{Functions}
\begin{verbatim}
`a         to_array<`a>(Xarray<`a>) [?];
Xarray<`a> from_array<`a>(`a arr[?]);
\end{verbatim}

\subsubsection*{Use}

\begin{defun}{to_array}{(a)}
Returns a new (non-extensible) array with the same elements as the
xarray \var{a}.
\end{defun}

\begin{defun}{from_array}{(b)}
Returns a new xarray with the same elements as the array \var{b}.
\end{defun}

\subsubsection*{Iteration on xarrays}
\subsubsection*{Functions}
\begin{verbatim}
void app<`a,`b>(`b f(`a), Xarray<`a>);
void app_c<`a,`b,`c>(`b f(`c,`a), `c, Xarray<`a>);
void iter<`a>(void f(`a), Xarray<`a>);
void iter_c<`a,`b>(void f(`b,`a), `b, Xarray<`a>);
\end{verbatim}

\subsubsection*{Use}

\begin{defun}{app}{(f,a)}
Applies the function \var{f} to every element of \var{a}, in order from
low to high index.  The results of the applications are discarded.
\end{defun}

\begin{defun}{app_c}{(f,env,a)}
A version of app where the function \var{f} takes a closure.
\end{defun}

\begin{defun}{iter}{(f,a)}
A version of app where the function \var{f} returns void.
\end{defun}

\begin{defun}{iter_c}{(f,env,a)}
A version of iter where the function \var{f} takes a closure.
\end{defun}

\subsubsection*{Mapping over xarrays}
\subsubsection*{Functions}
\begin{verbatim}
Xarray<`b> map<`a,`b>(`b f(`a), Xarray<`a>);
Xarray<`b> map_c<`a,`b,`c>(`b f(`c,`a), `c, Xarray<`a>);
\end{verbatim}

\subsubsection*{Use}

\begin{defun}{map}{(f,a)}
Returns an xarray with elements \var{f}(\var{x1})..\var{f}(xn), where
\var{x1}..\var{xn} are the elements of xarray \var{a}.
\end{defun}

\begin{defun}{map_c}{(f,env,a)}
A version of map where the function \var{f} takes a closure argument.
\end{defun}

\subsubsection*{Miscellaneous operations on xarrays}
\subsubsection*{Functions}
\begin{verbatim}
int length<`a>(Xarray<`a>);
void reuse<`a>(Xarray<`a> xarr);
\end{verbatim}

\subsubsection*{Use}

\begin{defun}{length}{(a)}
Returns the number of elements in the xarray \var{a}.
\end{defun}

\begin{defun}{reuse}{(a)}
Sets the number of elements in xarray \var{a} to 0.
\end{defun}

% Local Variables:
% TeX-master: "main-screen"
% End:
