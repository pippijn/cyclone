\section{Type Inference}
\hypertarget{type_inference_sec}{}

Cyclone allows many explicit types to be elided.  In short, you write
\texttt{_} (underscore) where a type should be and the compiler tries to
figure out the type for you.  Type inference can make C-like Cyclone
code easier to write and more readable.  For example,
\begin{verbatim}
  _ x = malloc(sizeof(sometype_t));
\end{verbatim}
is a fine substitute for
\begin{verbatim}
  sometype_t * x = malloc(sizeof(sometype_t));
\end{verbatim}
Of course, explicit types can make code more readable, so it is often
better style not to use inference.

Inference is even more useful because of Cyclone's advanced typing
constructs.  For example, it is much easier to write down \texttt{_}
than a type for a function pointer.  

We now give a rough idea of when you can elide types and how types get
inferred.  In practice, you tend to develop a sense of which idioms
succeed, and, if there's a strange compiler-error message about a
variable's type, you give more explicit information about the
variable's type.

\paragraph{Syntax}
As far as the parser is
concerned, \texttt{_} is a legal type specifier.  However, the type-checker
will immediately reject \texttt{_} in these places (or at least it
should):

\begin{itemize}
\item As part of a top-level variable or function's type.
\item As part of a \texttt{struct}, \texttt{union}, 
 \texttt{datatype}, or
  \texttt{typedef} declaration.
\end{itemize}

Note that \texttt{_} can be used for part of a type.  A silly example is
\texttt{\$(_,int) = \$(3,4)}; a more useful example is an explicit cast to
a non-nullable pointer (to avoid a compiler warning).  For example:
\begin{verbatim}
  void f(some_big_type * x, some_big_type @ y) {
  if(x != NULL) {
    y = (_ *@notnull) x;
  }
\end{verbatim}

\paragraph{Semantics}
Except for the subtleties discussed below, using \texttt{_} should not change the
meaning of programs.  However, it may cause a program not to type-check
because the compiler no longer has the type information it needs at some point
in the program.  For example, the compiler rejects \texttt{x->f} if it does not
know the type of \texttt{x} because the different struct types can have
members named \texttt{f}. 

The compiler infers the types of expressions based on uses.  For
example, consider:
\begin{verbatim}
  _ x = NULL;
  x = g();
  x->f;
\end{verbatim}
This code will type-check provided the return type of \texttt{g} is a
pointer to a struct with a field named \texttt{f}.  If the two
statements were in the other order, the code would not type-check. 
Also, if \texttt{g} returned an \texttt{int}, the code would not
type-check, even without the \texttt{x->f} expression, because the
\texttt{_ x = NULL} constrains \texttt{x} to have a pointer type.

However, the above discussion assumes that sequences of statements are
type-checked in order.  This is true, but \emph{in general the
  type-checker's order is unspecified}.

\paragraph{Subtleties}
In general, inference has subtle interactions with implicit coercions
(such as from \texttt{t*@notnull} to \texttt{t*@nullable}) and
constants that have multiple types (such as numeric constants).

The following is a desirable property: If a program is modified by
replacing some explicit types with \texttt{_} and the program still
type-checks, then its meaning is the same.  \emph{This property does
  not hold!}  Here are two examples:

\paragraph{Numeric Types} This program prints -24 1000:
\begin{verbatim}
  int f() {
  char c = 1000;
  return c;
  }
  int g() {
  _ c = 1000; // compiler infers int
  return c;
  }
  int main() {
  printf("%d %d", f(), g());
  return 0;
  }
\end{verbatim}

\paragraph{Order Matters}
Here is an example where the function's meaning depends on the order
the type-checker examines the function:
\begin{verbatim}
  void h1(int *@notnull c, int maybe) {
    _ a;
    if(maybe)
      a = c;
    else
      a = NULL;
  }
\end{verbatim}
At first, the type of \texttt{a} is completely unconstrained.  If we
next consider \texttt{a = c}, we will give \texttt{a} the type of
\texttt{c}, namely \texttt{int *@notnull}, an int pointer that cannot be
\texttt{NULL}.  Clearly that makes the assignment \texttt{a = NULL}
problematic, but Cyclone allows assignment from nullable pointers to
non-nullable pointers; it gives a compile-time warning and inserts a
run-time check that the value is not \texttt{NULL}.  Here the check will
fail and an exception will be raised.  That is, \texttt{h1(p,0)} is
guaranteed to raise an exception.

But what if the type-checker examines \texttt{a = NULL} first?  Then the
type-checker will constrain \texttt{a}'s type to be a nullable pointer
to an unconstrained type.  Then the assignment \texttt{a = c} will
constrain that type to be \texttt{int}, so the type of \texttt{a} is
\texttt{int *}.  An assignment from \texttt{int *@notnull} to \texttt{int *} is
safe, so there is no warning.  Moreover, the assignment \texttt{a = NULL} is
not a run-time error.

\emph{The order of type-checking is left unspecified.  In the future,
  we intend to move to a system that is order-independent.}

% Local Variables:
% TeX-master: "main-screen"
% End:
