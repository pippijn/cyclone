\section{Introduction}

Cyclone is a language for C programmers who want to write secure,
robust programs.  It's a dialect of C designed to be \emph{safe}: free
of crashes, buffer overflows, format string attacks, and so on.
Careful C programmers can produce safe C programs, but, in practice,
many C programs are unsafe.  Our goal is to make \emph{all} Cyclone
programs safe, regardless of how carefully they were written.  All
Cyclone programs must pass a combination of compile-time, link-time,
and run-time checks designed to ensure safety.

There are other safe programming languages, including Java, ML, and
Scheme.  Cyclone is novel because its syntax, types, and semantics are
based closely on C\@.  This makes it easier to interface Cyclone with
legacy C code, or port C programs to Cyclone.  And writing a new
program in Cyclone ``feels'' like programming in C: Cyclone tries to
give programmers the same control over data representations, memory
management, and performance that C has.

Cyclone's combination of performance, control, and safety make it a
good language for writing systems and security software.  Writing such
software in Cyclone will, in turn, motivate new research into safe,
low-level languages.  For instance, originally, all heap-allocated
data in Cyclone were reclaimed via a conservative garbage collector.
Though the garbage collector ensures safety by preventing programs
from accessing deallocated objects, it also kept Cyclone from being
used in latency-critical or space-sensitive applications such as
network protocols or device drivers.  To address this shortcoming, we
have added a region-based memory management system based on the work
of Tofte and Talpin.  The region-based memory manager allows you some
real-time control over memory management and can significantly reduce
space overheads when compared to a conventional garbage collector.
Furthermore, the region type system ensures the same safety properties
as a collector: objects cannot be accessed outside of their lifetimes.

This manual is meant to provide an informal introduction to Cyclone.
We have tried to write the manual from the perspective of a C 
programmer who wishes either to port code from C to Cyclone, or
develop a new system using Cyclone.  Therefore, we assume a fairly
complete understanding of C.

Obviously, Cyclone is a work in progress and we expect to make
substantial changes to the design and implementation.  Your feedback
(and patience) is greatly appreciated.

\subsection{Related Projects}

You might wonder what is the difference between the Cyclone project
and a few other projects, notably
\href{http://lclint.cs.virginia.edu/}{LCLint} from MIT and Virginia,
the Extended Static Checking (ESC) project from Compaq SRC, and the
\href{http://www.cs.wisc.edu/~austin/scc.html}{Safe-C} project from
the University of Wisconsin.  These are good systems and we have great
respect for the work that has gone into them.  But their goals are
quite different from Cyclone and we view our work as somewhat
overlapping, but mostly complementary.

Though we are interested in finding and eliminating bugs, we also want
to guarantee a certain level of safety. LCLint can't provide such a
guarantee, because the underlying language (C) and associated static
analyses are unsound (i.e., some bugs will not be caught). The reason
for this is that without changing C, it is extremely difficult to
guarantee safety without a lot of extra runtime overhead, or a lot of
false positives in your static analysis. LCLint is aimed at being a
practical tool for debugging C code, and as such, makes compromises to
eliminate false positives, but at the expense of soundness. Thus,
LCLint can't be used in settings where security is a concern.

ESC is similar in some ways to LCLint, but is built on top of Java. (A
previous version was built on top of Modula-3). Since Java and
Modula-3 are type-safe, there is a fundamental safety guarantee
provided by the ESC system. And in addition, one gets the extra
(really amazing) bug-finding power of ESC\@. However, as we argued
above, many systems programming tasks cannot be coded in languages
such as Java (or Modula-3) or at least can't be coded efficiently. In
addition, it is difficult to move legacy code written in C to these
environments.

Consequently, we see the ESC and LCLint projects as complementary to
the Cyclone project. Ideally, after Cyclone reaches a stable design
point, we should be able to put an ``ESC/Cyclone'' or
``LCLint/Cyclone'' static debugging tool together.

The Safe-C project out of Wisconsin is perhaps the closest in spirit
to ours, in that they are attempting to provide a memory-safety
guarantee. However, the approach taken by Safe-C is quite different
than Cyclone: Programs continue to be written in C, but extra
indirections and dynamic checks are inserted to ensure that
abstractions are respected. For example, pointers in Safe-C are
actually rather complex data structures, and dereferencing a pointer
is an expensive operation as a number of checks must be done to ensure
that the pointer is still valid. In order to avoid some of these
overheads, the Wisconsin folks apply sophisticated, automatic
analyses.

The advantage of Safe-C is that, in principle, programs can be compiled
and run with no modification by the programmer. The disadvantages are
that (a) there is still a great deal of space and time overhead for
programs, (b) the data representations change radically. As such, the
whole system (e.g., the standard libraries) has to be recompiled and
pieces of code cannot be migrated from C to Safe-C\@. In contrast,
Cyclone goes to great lengths to preserve the data representations so
that Cyclone and C code can smoothly interoperate. Furthermore, many
more checks are performed statically in Cyclone. The downside is that
programmers have to change the code (typically, by modifying or adding
more typing information), and they can't always write what they want.
Our research is aimed at eliminating these drawbacks and to compare
against the Safe-C approach.

\subsection{Acknowledgements}

The people involved in the development of Cyclone are at Cornell and
AT\&T\@.  Dan Grossman, Trevor Jim, and Greg Morrisett worked out the
initial design and implementation, basing the language to some degree
on Popcorn, a safe-C-like language that was developed at Cornell as
part of the
\href{http://www.cs.cornell.edu/talc}{Typed Assembly Language} (TAL)
project.  Mathieu Baudet contributed the bulk of the code for the
link-checker.  Matthew Harris did much of the hard work needed to
wrap and import the necessary libraries.  Yanling Wang ported bison
to Cyclone.  All of these people have also contributed by finding
and fixing various bugs.  A number of other people have also helped
to find bugs and/or contributed key design ideas including James Cheney, 
Fred Smith, Nathan Lutchansky, Jeff Vinocur, and David Walker.  

% Local Variables:
% TeX-master: "main-screen"
% End:
