\section{Porting C code to Cyclone}

%HEVEA \begin{latexonly}
\ifscreen
\newenvironment{porta}[2]{%
  \begin{list}{}{}%
  \item[\hypertarget{#1}{\colorbox{lightblue}{\textbf{#2}}}]}{\end{list}}
\else
\newenvironment{porta}[2]{%
  \begin{list}{}{}%
  \item[\hypertarget{#1}{\textbf{#2}}]}{\end{list}}
\fi
%HEVEA \end{latexonly}
%HEVEA \newenvironment{porta}[2]{%
%HEVEA   \begin{list}{}{}%
%HEVEA   \item[\hypertarget{#1}{%
%HEVEA \begin{rawhtml}<table><tr><td bgcolor="c0d0ff">\end{rawhtml}%
%HEVEA \textbf{#2}%
%HEVEA \begin{rawhtml}</td></tr></table>\end{rawhtml}%
%HEVEA }]}{\end{list}}
Though Cyclone resembles and shares a lot with C, porting is not
always straightforward.  Furthermore, it's rare that you actually port
an entire application to Cyclone.  You may decide to leave certain
libraries or modules in C and port the rest to Cyclone.  In this
chapter, we want to share with you the tips and tricks that we have
developed for porting C code to Cyclone and interfacing Cyclone
code against legacy C code.

\subsection{Semi-Automatic Porting}

The Cyclone compiler includes a simple porting mode which you can
use to try to move your C code closer to Cyclone.  The porting
tool is not perfect, but it's a start and we hope to develop it
more in the future.  

When porting a file, say \texttt{foo.c}, you'll first need to
copy the file to \texttt{foo.cyc} and then edit it to
add \texttt{__cyclone_port_on__;} and 
\texttt{__cyclone_port_off__;} around the code that you want
Cyclone to port.  For example, if after copying \texttt{foo.c},
the file \texttt{foo.cyc} contains the
following:
\begin{verbatim}
1.   #include <stdio.h>
2. 
3.   void foo(char *s) {
4.     printf(s);
5.   }
6. 
7.   int main(int argc, char **argv) {
8.     argv++;
9.     for (argc--; argc >= 0; argc--, argv++)
10.      foo(*argv);
11.  }
\end{verbatim}
then you'll want to insert \texttt{__cyclone_port_on__;} at line
2 and \texttt{__cyclone_port_off__;} after line 11.  You do not
want to port standard include files such as \texttt{stdio}, hence
the need for the delimiters.

Next compile the file with the \texttt{-port} flag:
\begin{verbatim}
  cyclone -port foo.cyc > rewrites.txt
\end{verbatim}
and pipe the output to a file, in this case \texttt{rewrites.txt}.
If you edit the output file, you will see that the compiler has
emitted a list of edits such as the following:
\begin{verbatim}
  foo.cyc(5:14-5:15): insert `?' for `*'
  foo.cyc(9:24-9:25): insert `?' for `*'
  foo.cyc(9:25-9:26): insert `?' for `*'
\end{verbatim}
You can apply these edits by running the \texttt{rewrite} program
on the edits:
\begin{verbatim}
  rewrite -port foo.cyc > rewrites.txt
\end{verbatim}
(The \texttt{rewrite} program is written in Cyclone and included
in the \texttt{tools} sub-directory.)  This will produce a new
file called \texttt{foo_new.cyc} which should look like this:
\begin{verbatim}
#include <stdio.h>

__cyclone_port_on__;

void foo(char ?s) { 
  printf(s);
}

int main(int argc, char ??argv) {
  argv++;
  for (argc--; argc >= 0; argc--, argv++) 
    foo(*argv);
}
__cyclone_port_off__;
\end{verbatim}
Notice that the porting system has changed the pointers from
thin pointers to fat pointers (\texttt{?}) to support the pointer
arithmetic that is done in \texttt{main}, and that this 
constraint has flowed to procedures that are called (e.g., \texttt{foo}).

You'll need to strip out the port-on and port-off directives and
then try to compile the file with the Cyclone compiler.  In this
case, the rewritten code in \texttt{foo_new.cyc} compiles with
a warning that \texttt{main} might not return an integer value.
In general, you'll find that the porting tool doesn't always
produce valid Cyclone code.  Usually, you'll have to go in and
modify the code substantially to get it to compile.  Nonetheless,
the porting tool can take care of lots of little details for
you.  

\subsection{Manually Translating C to Cyclone}

To a first approximation, you can port a simple program
from C to Cyclone by following these steps which are
detailed below:
\begin{itemize}
\item \hyperlink{port:pointers}{Change pointer types to fat pointer types where necessary.}  
\item \hyperlink{port:malloc}{Use comprehensions to heap-allocate arrays.}
\item \hyperlink{port:unions}{Use tagged unions for unions with pointers.}
\item \hyperlink{port:initialize}{Initialize variables.}  
\item \hyperlink{port:cases}{Put breaks or fallthrus in switch cases.} 
\item \hyperlink{port:temp}{Replace one temporary with multiple temporaries.}  
\item \hyperlink{port:rgns}{Connect argument and result pointers with the same region.}
\item \hyperlink{port:types}{Insert type information to direct the type-checker.}
\item \hyperlink{port:const}{Copy ``const'' code or values to make it non-const.}  
\item \hyperlink{port:free}{Get rid of calls to free, calloc, etc.}
\item \hyperlink{port:poly}{Use polymorphism or tagged unions to get rid of void*.}  
\item \hyperlink{port:vararg}{Rewrite the bodies of vararg functions.}
\item \hyperlink{port:setjmp}{Use exceptions instead of setjmp.}
\end{itemize}

Even when you follow these suggestions, you'll still need to test and
debug your code carefully.  By far, the most common run-time errors
you will get are uncaught exceptions for null-pointer dereference
or array out-of-bounds.  Under Linux, you should get a stack backtrace
when you have an uncaught exception which will help narrow down
where and why the exception occurred.  On other architectures, you
can use \texttt{gdb} to find the problem.  The most effective way
to do this is to set a breakpoint on the routines \texttt{\_throw\_null()}
and \texttt{\_throw\_arraybounds()} which are defined in the
runtime and used whenever a null-check or array-bounds-check fails.
Then you can use \texttt{gdb}'s backtrace facility to see where
the problem occurred.  Of course, you'll be debugging at the C
level, so you'll want to use the \texttt{-save-c} and \texttt{-g}
options when compiling your code.  

\begin{porta}{port:pointers}{Change pointer types to fat pointer types where necessary.}  
Ideally, you should examine the code and use thin pointers (e.g., \texttt{int*}
or better \texttt{int*@notnull}) wherever possible as these require fewer
run-time checks and less storage.  However, recall that thin pointers
do not support pointer arithmetic.  In those situations, you'll need
to use fat pointers (e.g., \texttt{int*@fat} which can also be written
as \texttt{int?}).  A particularly simple strategy 
when porting C code is to just change all pointers to fat pointers.
The code is then more likely to compile, but will have greater overhead.
After changing to use all fat pointers, you may wish to profile or reexamine
your code and figure out where you can profitably use thin pointers.

Be careful with \texttt{char} pointers.  By default, a \texttt{char ?}
is treated as zero-terminated, i.e. a \texttt{char * @fat @zeroterm}.
If you are using the \texttt{char} pointer as a buffer of bytes, then
you may actually wish to change it to be a \texttt{char ? @nozeroterm}
instead.  Along these lines, you have to be careful that when you are
using arrays that get promoted to pointers, that you correctly
indicate the size of the array to account for the zero terminator.
For example, say your original C code was
\begin{verbatim}
    char line[MAXLINELEN];
    while (fgets(line, MAXLINELEN, stdin)) ...
\end{verbatim}
If you want your pointer to be zero-terminated, you would have to do
the following:
\begin{verbatim}
    char line[MAXLINELEN+1] @zeroterm;
    while (fgets(line, MAXLINELEN, stdin)) ...
\end{verbatim}
The \texttt{@zeroterm} qualifier is needed since \texttt{char}
\emph{arrays} are not zero-terminated by default.  Adding the
\texttt{+1} makes space for the extra zero terminator that Cyclone
includes, ensuring that it won't be overwritten by \texttt{fgets}.  If
you don't do this, you could well get an array bounds exception at
runtime.  If you don't want your \texttt{char} array to be
zero-terminated, you can simply leave the original C code as is.
\end{porta}

\begin{porta}{port:malloc}{Use comprehensions to heap-allocate arrays.}
Cyclone provides limited support for \texttt{malloc} and separated
initialization but this really only works for bits-only objects.
To heap- or region-allocate and initialize an array that might contain
pointers, use
\texttt{new} or \texttt{rnew} in conjunction with array comprehensions.  
For example, to copy a vector of integers \texttt{s}, one might write:
\begin{verbatim}
  int *@fat t = new {for i < numelts(s) : s[i]};
\end{verbatim}
\end{porta}

\begin{porta}{port:unions}{Use tagged unions for unions with pointers.}
Cyclone only lets you read members of unions that 
contain ``bits'' (i.e., ints; chars;
shorts; floats; doubles; or tuples, structs, unions, or arrays of bits.)
So if you have a C union with a pointer type in it, you'll have to
code around it.  One way is to simply use a \texttt{@tagged} union.
Note that this adds hidden tag and associated checks to ensure safety.
\end{porta}

\begin{porta}{port:initialize}{Initialize variables.}  
Top-level variables must be initialized
in Cyclone, and in many situations, local variables must be initialized.
Sometimes, this will force you to change the type of the variable
so that you can construct an appropriate initial value.  For instance,
suppose you have the following declarations at top-level:
\begin{verbatim}
struct DICT; 
struct DICT *@notnull new_dict();
struct DICT *@notnull d;
void init() {
  d = new_dict();
}
\end{verbatim}

Here, we have an abstract type for dictionaries 
(\texttt{struct Dict}), a constructor
function (\texttt{new\_dict()}) which returns a pointer to a new
dictionary, and a top-level variable (\texttt{d}) which is meant
to hold a pointer to a dictionary.  The \texttt{init} function
ensures that \texttt{d} is initialized.  However, 
Cyclone would complain that 
\texttt{d} is not initialized because \texttt{init} may not be
called, or it may only be called after \texttt{d} is already used.  
Furthermore, the only way to initialize \texttt{d}
is to call the constructor, and such an expression is not a 
valid top-level initializer.  The solution is to declare \texttt{d} as
a ``possibly-null'' pointer to a dictionary and initialize it
with \texttt{NULL}:
\begin{verbatim}
struct DICT; 
struct DICT *nonnull new_dict();
struct DICT *d;
void init() {
  d = new_dict();
}
\end{verbatim}

Of course, now whenever you use \texttt{d}, either you or the compiler
will have to check that it is not \texttt{NULL}.
\end{porta}

\begin{porta}{port:cases}{Put breaks or fallthrus in switch cases.}  
Cyclone requires
that you either break, return, continue, throw an exception, or explicitly
fallthru in each case of a switch.  
\end{porta}

\begin{porta}{port:temp}{Replace one temporary with multiple temporaries.}
Consider the following code:
\begin{verbatim}
void foo(char * x, char * y) {
  char * temp;
  temp = x;
  bar(temp);
  temp = y;
  bar(temp);
}
\end{verbatim}

When compiled, Cyclone generates an error message like this:
\begin{verbatim}
type mismatch: char *@zeroterm #0  != char *@zeroterm #1 
\end{verbatim}

The problem is that Cyclone thinks that \texttt{x} and \texttt{y}
might point into different regions (which it named \texttt{\#0} and
\texttt{\#1} respectively), and the variable \texttt{temp} is assigned
both the value of \texttt{x} and the value of \texttt{y}.  Thus,
there is no single region that we can say \texttt{temp} points into.
The solution in this case is to use two different temporaries for
the two different purposes:
\begin{verbatim}
void foo(char * x, char * y) {
  char * temp1;
  char * temp2;
  temp1 = x;
  bar(temp1);
  temp2 = y;
  bar(temp2);
}
\end{verbatim}

Now Cyclone can figure out that \texttt{temp1} is a pointer into
the region \texttt{\#0} whereas \texttt{temp2} is a pointer into
region \texttt{\#1}.  
\end{porta}

\begin{porta}{port:rgns}{Connect argument and result pointers with the same region.}
Remember that Cyclone assumes that pointer inputs to a function might
point into distinct regions, and that output pointers, by default point
into the heap.  Obviously, this won't always be the case.  Consider
the following code:
\begin{verbatim}
int *foo(int *x, int *y, int b) {
  if (b)
    return x;
  else
    return y;
}
\end{verbatim}

Cyclone complains when we compile this code:
\begin{verbatim}
foo.cyc:3: returns value of type int *`GR0 but requires int *
  `GR0 and `H are not compatible. 
foo.cyc:5: returns value of type int *`GR1 but requires int *
  `GR1 and `H are not compatible. 
\end{verbatim}
The problem is that neither \texttt{x} nor \texttt{y} is a pointer
into the heap.  You can fix this problem by putting in explicit regions
to connect the arguments and the result.  For instance, we might write:
\begin{verbatim}
int *`r foo(int *`r x, int *`r y, int b) {
  if (b)
    return x;
  else
    return y;
}
\end{verbatim}
and then the code will compile.  Of course, any caller to this function
must now ensure that the arguments are in the same region.  
\end{porta}

\begin{porta}{port:types}{Insert type information to direct the type-checker.}
Cyclone is usually good about inferring types.  But sometimes, it
has too many options and picks the wrong type.  A good example is
the following:
\begin{verbatim}
void foo(int b) {
  printf("b is %s", b ? "true" : "false");
} 
\end{verbatim}

When compiled, Cyclone warns:
\begin{verbatim}
(2:39-2:40): implicit cast to shorter array
\end{verbatim}

The problem is that the string \texttt{"true"} is assigned the
type \texttt{const char ?\{5\}} whereas the string
\texttt{"false"} is assigned the type \texttt{const char ?\{6\}}.
(Remember that string constants have an implicit 0 at the end.)
The type-checker needs to find a single type for both since
we don't know whether \texttt{b} will come out true or false
and conditional expressions require the same type for either
case.  There are at least two ways that the types of the strings can be
promoted to a unifying type.  One way is to promote both
to \texttt{char?} which would be ideal.  Unfortunately, Cyclone
has chosen another way, and promoted the longer string
(\texttt{"false"}) to a shorter string type, namely
\texttt{const char ?\{5\}}.  This makes the two types the
same, but is not at all what we want, for when the procedure
is called with false, the routine will print
\begin{verbatim}
b is fals
\end{verbatim}

Fortunately, the warning indicates that there might be a problem.
The solution in this case is to explicitly cast at least one of the two
values to \texttt{const char ?}:
\begin{verbatim}
void foo(int b) {
  printf("b is %s", b ? ((const char ?)"true") : "false");
} 
\end{verbatim}

Alternatively, you can declare a temp with the right type and use
it:
\begin{verbatim}
void foo(int b) {
  const char ? t = b ? "true" : "false"
  printf("b is %s", t);
} 
\end{verbatim}

The point is that by giving Cyclone more type information, you can
get it to do the right sorts of promotions.  Other sorts of type
information you might provide include region annotations (as outlined
above), pointer qualifiers, and casts.
\end{porta}

\begin{porta}{port:const}{Copy ``const'' code or values to make it non-const.}
Cyclone takes \texttt{const} seriously.  C does not.  Occasionally,
this will bite you, but more often than not, it will save you from
a core dump.  For instance, the following code will seg fault on
most machines:
\begin{verbatim}
void foo() {
  char ?x = "howdy"
  x[0] = 'a';
}
\end{verbatim}

The problem is that the string \texttt{"howdy"} will be placed in
the read-only text segment, and thus trying to write to it will
cause a fault.  Fortunately, Cyclone complains that you're trying
to initialize a non-const variable with a const value so this
problem doesn't occur in Cyclone.  If you really want to initialize
\texttt{x} with this value, then you'll need to copy the string,
say using the \texttt{dup} function from the string library:
\begin{verbatim}
void foo() {
  char ?x = strdup("howdy");
  x[0] = 'a';
}
\end{verbatim}

Now consider the following call to the \texttt{strtoul} code in the
standard library:
\begin{verbatim}
extern unsigned long strtoul(const char ?`r n, 
                             const char ?`r*`r2 endptr,
                             int base);
unsigned long foo() {
  char ?x = strdup("howdy");
  char ?*e = NULL;
  return strtoul(x,e,0);
}
\end{verbatim}

Here, the problem is that we're passing non-const values to the
library function, even though it demands const values.  Usually,
that's okay, as \texttt{const char ?} is a super-type of
\texttt{char ?}.  But in this case, we're passing as the
\texttt{endptr} a pointer to a \texttt{char ?}, and it
is not the case that \texttt{const char ?*} is a super-type
of \texttt{char ?*}.  In this case, you have two options:
Either make \texttt{x} and \texttt{e} const, or copy the
code for \texttt{strtoul} and make a version that doesn't
have const in the prototype.  
\end{porta}

\begin{porta}{port:free}{Get rid of calls to free, calloc etc.}
There are many standard functions that Cyclone can't support 
and still maintain type-safety.  An obvious one is \texttt{free()}
which releases memory.  Let the garbage collector free the object
for you, or use region-allocation if you're scared of the collector.
Other operations, such as \texttt{memset}, \texttt{memcpy}, 
and \texttt{realloc} are supported, but in a limited fashion in 
order to preserve  type safety.  
\end{porta}

\begin{porta}{port:poly}{Use polymorphism or tagged unions to get rid of void*.}  
Often you'll find C code that uses \texttt{void*} to simulate
polymorphism.  A typical example is something like swap:
\begin{verbatim}
void swap(void **x, void **y) {
  void *t = x;
  x = y;
  y = t;
}
\end{verbatim}

In Cyclone, this code should type-check but you won't be able
to use it in many cases.  The reason is that while \texttt{void*}
is a super-type of just about any pointer type, it's not the
case that \texttt{void**} is a super-type of a pointer to a
pointer type.  In this case, the solution is to use Cyclone's
polymorphism:
\begin{verbatim}
void swap(`a *x, `a *y) {
  `a t = x;
  x = y;
  y = t;
}
\end{verbatim}

Now the code can (safely) be called with any two (compatible)
pointer types.  This trick works well as long as you only need
to ``cast up'' from a fixed type to an abstract one.  It doesn't
work when you need to ``cast down'' again.  For example, consider
the following:
\begin{verbatim}
int foo(int x, void *y) {
  if (x)
   return *((int *)y);
  else {
    printf("%s\n",(char *)y);
    return -1;
  }
}
\end{verbatim}

The coder intends for \texttt{y} to either be an int pointer or
a string, depending upon the value of \texttt{x}.  If \texttt{x}
is true, then \texttt{y} is supposed to be an int pointer, and
otherwise, it's supposed to be a string.  In either case, you have
to put in a cast from \texttt{void*} to the appropriate type,
and obviously, there's nothing preventing someone from passing
in bogus cominations of \texttt{x} and \texttt{y}.  The solution
in Cylcone is to use a tagged union to represent the dependency
and get rid of the variable \texttt{x}:
\begin{verbatim}
@tagged union IntOrString { 
  int Int;
  const char *@fat String;
};
typedef union IntOrString i_or_s;
int foo(i_or_s y) {
  switch (y) {
  case {.Int = i}:  return i;
  case {.String = s}:  
    printf("%s\n",s);
    return -1;
  }
}
\end{verbatim}
\end{porta}

\begin{porta}{port:vararg}{Rewrite the bodies of vararg functions.}
See the section on varargs for more details.  
\end{porta}

\begin{porta}{port:setjmp}{Use exceptions instead of setjmp.}
Many uses of \texttt{setjmp}/\texttt{longjmp} can be replaced
with a \texttt{try}-block and a \texttt{throw}.  Of course,
you can't do this for things like a user-level threads package,
but rather, only for those situations where you're trying
to ``pop-out'' of a deeply nested set of function calls.
\end{porta}

\subsection{Interfacing to C}
When porting any large code from C to Cyclone, or even when writing
a Cyclone program from scratch, you'll want to be able to access
legacy libraries.  To do so, you must understand how Cyclone
represents data structures, how it compiles certain features,
and how to write wrappers to make up for representation mismatches.

\subsubsection{Extern ``C''}

Sometimes, interfacing to C code is as simple as writing
an appropriate interface.  For instance, if you want to
call the \texttt{acos} function which is defined in the C
Math library, you can simply write the following:
\begin{verbatim}
  extern "C" double acos(double);
\end{verbatim}

The \texttt{extern "C"} scope declares that the function is
defined externally by C code.  As such, it's name is not
prefixed with any namespace information by the compiler.
Note that you can still embed the function within a Cyclone
namespace, it's just that the namespace is ignored by the
time you get down to C code.  
If you have a whole group of functions then you can wrap them with
a single \texttt{extern "C" \{ ... \}}, as in:
\begin{verbatim}
  extern "C" {
    double acos(double);
    float  acosf(float);
    double acosh(double);
    float  acoshf(float);
    double asin(double);
  }
\end{verbatim}
You must be careful that the type you declare for the C function is
its real type.  Misdeclaring the type could result in a runtime
error.  Note that you can add Cyclonisms to the type that refine the
meaning of the original C.  For example, you could declare:
\begin{verbatim}
  extern "C" int strlen(const char * @notnull str);
\end{verbatim}
Here we have refined the type of \texttt{strlen} to require that a
non-NULL pointer is passed to it.  Because this type is
representation-compatible with the C type (that is, it has the same
storage requirements and semantics), this is legal.  However, the
following would be incorrect:
\begin{verbatim}
  extern "C" int strlen(const char * @fat str);
\end{verbatim}
Giving the function this type would probably lead to an error because
Cyclone fat pointers are represented as three words, but the standard
C library function expects a single pointer (one word).

The \texttt{extern "C"} approach works well enough that it covers many
of the cases that you'll encounter.  However, the situation is not so
when you run into more complicated interfaces.  Sometimes you will
need to write some wrapper code to convert from Cyclone's
representations to C's and back (so called \emph{wrapper} code).

\subsubsection{Extern ``C include''}

Another useful tool is the \texttt{extern "C include"} mechanism.
It allows you to write C definitions within a Cyclone file.  Here
is a simple example:
\begin{verbatim}
extern "C include" {
  char peek(unsigned int i) {
    return *((char *)i);
  }

  void poke(unsigned int i, char c) {
    *((char *)i) = c;
  }
} export {
  peek, poke;
}
\end{verbatim}
In this example, we've defined two C functions \texttt{peek} and
\texttt{poke}.  Cyclone will not compile or type-check their code, but
rather pass them on to the C compiler.  The \texttt{export} clause
indicates which function and variable definitions should be exported
to the Cyclone code.  If we only wanted to export the \texttt{peek}
function, then we would leave the \texttt{poke} function out of the
export list.  All all other definitions, like \texttt{typedef}s,
\texttt{struct}s, etc., not to mention \texttt{\#define}s and other
preprocessor effects, are exported by default (but this may change in
a later release).

Any top-level types you mention in the \texttt{extern "C include"} are
interpreted by the Cyclone code that uses them as Cyclone types.  If
they are actually C types (as would be the case if you
\texttt{\#include}d some header in the C block), this will be safe, but
possibly undesirable, since they may not communicate the right
information  to the Cyclone code.  There are two ways around this.
In many cases, you can actually declare Cyclone types within the C
code, and they will be treated as such.  For example, in
\texttt{lib/core.cyc}, we have
For example, you could do something like:
\begin{verbatim}
extern "C include" {
  ... Cyc_Core_mkthin(`a ?`r dyn, sizeof_t<`a> sz) {
    unsigned bd = _get_dyneither_size(dyn,sz);
    return Cyc_Core_mktuple(dyn.curr,bd);
  } 
} export {
  Cyc_Core_mkthin
}
\end{verbatim}
In this case, we are able to include a \texttt{?} notation directly in
the C type, but then manipulate it using the runtime system functions
for fat pointers (see \texttt{cyc_include.h} for details).

In the case that you are \texttt{\#include}ing a C header file, you
may not be able to change its definitions to have a proper Cyclone
type, or it may be that the Cyclone definitions will not parse for
some reason.  In this case, you can declare a block to \emph{override}
the definitions with Cyclone compatible versions.  For example, we
could change the above code to be instead:
\begin{verbatim}
extern "C include" {
  struct foo { int x; int y; };
  struct foo *cast_charbuf(char *buf, unsigned int n) {
    if (n >= sizeof(struct foo))
      return (struct foo *)buf;
    else
      return (void *)0;
  }
} cyclone_override {
  struct foo *cast_charbuf
    (char * @numelts(valueof(`n)) @nozeroterm buf,tag_t<`n> n);
} export {
  cast_charbuf
}
\end{verbatim}
Now we have given \texttt{cast_charbuf} its original C type, but then
provided the Cyclone type in the override block.  The Cyclone type
ensures the value of \texttt{n} correctly represents the length of the
buffer, by using Cyclone's dependent types (see
\sectionref{sec:pointers}).  Note that top-level \texttt{struct} and
other type definitions can basically be entirely Cyclone syntax.  If
you try to declare a Cyclone overriding type that is
representation-incompatible with the C version, the compiler will
complain.

Here is a another example using an external header:
\begin{verbatim}
extern "C include" {  /* tell Cyclone that <pcre.h> is C code */
#include <pcre/pcre.h>
} cyclone_override {
  pcre *`U pcre_compile(const char @pattern, int options,
                        const char *`H *errptr, int *erroffset,
                        const unsigned char *tableptr);
  int pcre_exec(const pcre @code, const pcre_extra *extra, 
                const char *subject, int length,
                int startoffset, int options,
                int *ovector, int ovecsize);
} export { pcre_compile, pcre_exec; }
\end{verbatim}
In this case, we have included the Perl regular expression library C
header, and then exported two of its functions, \texttt{pcre_compile}
and \texttt{pcre_exec}.  Moreover, we have given these functions
Cyclone types that are more expressive in the original C.  Probably we
would yet want to write wrappers around these functions to check other
invariants of the arguments (e.g., that the \texttt{length} passed to
\texttt{pcre_exec} is indeed the length of the \texttt{subject}).
Take a look at \texttt{tests/pcredemo.cyc} for more information on
this example.  Another example that shows how you can override things
is in \texttt{tests/cinclude.cyc}.

The goal of this example is to show how you can safely suck in
a large C interface (in this case, the Perl Compatible Regular Expression
interface), write wrappers around some of the functions to convert
represenations and check properties, and then safely export these
wrappers to Cyclone.  

One word of warning: when you \texttt{\#include} something within an
\texttt{extern "C include"}, it will follow the normal include path,
which is to say that it will look for \emph{Cyclone} versions of the
headers first.  This means that if you do something like:
\begin{verbatim}
extern "C include" {
#include <string.h>
} export { ... }
\end{verbatim}
It will actually include the Cyclone version of the \texttt{string.h}
header!  These easiest way around this is to use an absolute path, as
in
\begin{verbatim}
extern "C include" {
#include "/usr/include/string.h"
} export { ... }
\end{verbatim}
Even worse is when a C header you wish to include itself includes a
header for which there exists a Cyclone version.  In the
\texttt{pcre.h} example above, this actually occurs in that
\texttt{pcre.h} includes \texttt{stdlib.h}, and gets the Cyclone
version.  To avoid this, the \texttt{pcredemo.cyc} program includes
the Cyclone versions of these headers first.  Ultimately we will
probably change the compiler so that header processing within
\texttt{extern "C include"} searches the C header path but not the
Cyclone one.
