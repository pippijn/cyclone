\section{Installing Cyclone}
\label{sec:install}
Cyclone currently only runs on 32-bit machines, and has only been
tested on Win32 (Cygnus) and Linux (Red Hat 6.2) platforms.  Other
platforms might or might not work.  Right now, there are a few 32-bit
dependencies in the compiler, so the system will probably not work on
a 64-bit machine without some changes.

To install and use Cyclone, you'll need to use the Gnu utilities,
including GCC (the Gnu C compiler) and Gnu-Make.  For Win32, you
should first install the latest version of the
\href{http://cygwin.com/}{Cygwin} utilities to do the build, and make
sure that the Cygwin \texttt{bin} directory is on your path. We use
some features of GCC extensively, so Cyclone definitely will not build
with another C compiler.

The file cyclone.zip (or cyclone.tar.gz) contains all of the files
that you need to build and install the system.  Unzip the files into a
directory; if you are installing Cyclone on a Windows system, we
suggest you choose \texttt{c:/cyclone}.  Once you've created the
directory, enter it and type ``\texttt{make}.''  This will create
executables for the Cyclone compiler (\texttt{cyclone} in Unix
systems, or \texttt{cyclone.exe} in Windows systems) and some
utilities in the \texttt{bin} subdirectory.

To use the Cyclone compiler, you should add the \texttt{bin}
subdirectory to your \texttt{PATH}.  In addition, if you have not
installed Cyclone in \texttt{c:/cyclone}, you must set the environment
variable \texttt{CYCLONE_INCLUDE_PATH} to
\textit{cycdir}\texttt{/include}, and the variable
\texttt{CYCLONE_EXEC_PREFIX} to \textit{cycdir}\texttt{/bin/cyc-lib},
where \textit{cycdir} is the directory where you installed Cyclone.

\subsection*{Bootstrapping}

If you're interested in working on the compiler itself, you'll want to
know more about how the sources are organized.  The Cyclone directory
has the following subdirectories:
\begin{center}
\begin{tabular}{ll}
\texttt{bin/}        & executables\\
~~\texttt{genfiles/} & generated C files for making the executables\\
\texttt{doc/}        & documentation\\
\texttt{include/}    & .h files for the libraries\\
\texttt{lib/}        & Cyclone source code for the libraries\\
\texttt{src/}        & Cyclone source code for the compiler\\
\texttt{tests/}      & some test code written in Cyclone\\
\texttt{tools/}      & source code of the lexer and parser generators
\end{tabular}
\end{center}

The Cyclone compiler is written in Cyclone, but since you won't have a
Cyclone compiler to start with, we have a directory,
\texttt{bin/genfiles}, that contains C files generated from our
Cyclone source files.  During the installation, these are compiled by
GCC to produce the executables in \texttt{bin}.  Once this is done,
the resulting Cyclone compiler can be used to compile the compiler
written in Cyclone.

To do this, simply type ``\texttt{make cyclone_src}.''  This will use
the Cyclone compiler to compile all of the files in the \texttt{lib}
and \texttt{src} directories (it will not overwrite the binaries in
the \texttt{bin/} directory).

If you want to make a change to the compiler and then start using the
change, you can type ``\texttt{make update}'' which will update the
\texttt{genfiles/} with any different C files that you generated.
Then you can type ``\texttt{make cyclone}'' to bootstrap with the new
code.

% Local Variables:
% TeX-master: "main"
% End:
