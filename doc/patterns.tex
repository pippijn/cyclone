\section{Pattern Matching}
\label{sec:patterns}

Pattern matching provides a concise, convenient way to bind parts of
large objects to new local variables.  Two Cyclone constructs use
pattern matching, let declarations and switch statements.  Although
the latter are more common, we first explain patterns with
\hyperlink{let_decls}{let declarations} because they have fewer
complications.  Then we describe all the \hyperlink{pat_forms}{pattern
  forms}.  Then we describe \hyperlink{switch_stmt}{switch
  statements}.

You must use patterns to access values carried by
\hyperlink{tagged_unions}{tagged unions}, including exceptions.  In
other situations, patterns make code more readable and less verbose.

\subsection{Let Declarations}\hypertarget{let_decls}{}

In Cyclone, you can write
\begin{verbatim}
  let x = e;
\end{verbatim}
as a local declaration.  The meaning is the same as \texttt{t x = e;}
where \texttt{t} is the type of \texttt{e}.  In other words, \texttt{x} is
bound to the new variable.  Unlike with variable declarations, a let
declaration must have an initializer (the \texttt{e}). Patterns are much
more powerful because they can bind several variables to different
parts of an aggregate object.  Here is an example:
\begin{verbatim}
  struct Pair {  int x; int y; };
  void f(struct Pair pr) {
    let Pair(fst,snd) = pr;
    ...
  }
\end{verbatim}

The pattern has the same structure as a \texttt{struct Pair} with parts
being variables.  Hence the pattern is a match for \texttt{pr} and the
variables are initialized with the appropriate parts of \texttt{pr}.  Hence
``\texttt{let Pair(fst,snd) = pr}'' is equivalent to
``\texttt{int fst =pr.x; int snd = pr.y}''.  A let-declaration's
initializer is evaluated only once.

Patterns may be as structured as the expressions against which they
match.  For example, given type
\begin{verbatim}
  struct Quad { struct Pair p1; struct Pair p2; };
\end{verbatim}
patterns for matching against an expression of type struct Quad could
be any of the following (and many more because of constants and
wildcards---see below):

\begin{itemize}
\item \texttt{Quad(Pair(a,b),Pair(c,d))}
\item \texttt{Quad(p1, Pair(c,d))}
\item \texttt{Quad(Pair(a,b), p2)}
\item \texttt{Quad(p1,p2)}
\item \texttt{q}
\end{itemize}

In general, a let-declaration has the form ``let p = e;'' where p is a
pattern and \texttt{e} is an expression.  In our example, the match
always succeeds, but in general patterns can have compile-time errors
or run-time errors.

At compile-time, the type-checker ensures that the pattern makes sense
for the expression.  For example, it rejects ``let Pair(fst,snd) = 0''
because 0 has type int but the pattern only makes sense for type
\texttt{struct Pair}.

Certain patterns are type-correct, but they may not match run-time
values.  For example, constants can appear in patterns, so ``let
Pair(17,snd) = pr;'' would match only when \texttt{pr.x} is 17.
Otherwise the exception \texttt{Match_Exception} is thrown.  Patterns
that may fail are rarely useful and poor style in let-declarations;
the compiler emits a warning when you use them.  In switch statements,
possibly-failing patterns are the norm---as we explain below, the
whole point is that one of the cases' patterns should match.

\subsection{Pattern Forms}\hypertarget{pat_forms}{}

So far, we have seen three pattern forms, variables patterns, struct
patterns, and constant patterns.  We now describe all the pattern
forms.  For each form, you need to know:

\begin{itemize}
\item The syntax
\item The types of expressions it can match against (to avoid a
  compile-time error)
\item The expressions the pattern matches against (other expressions
  cause a match failure)
\item The bindings the pattern introduces, if any.
\end{itemize}
There is one compile-time rule that is the same for all forms: All
variables (and type variables) in a pattern must be distinct.  For
example, ``let Pair(fst,fst) = pr;'' is not allowed.

You may want to read the descriptions for variable and struct patterns
first because we have already explained their use informally.

\begin{itemize}
\item \textbf{Variable patterns}
  \begin{itemize}
  \item Syntax: an identifer
  \item Types for match: all types
  \item Expressions matched: all expressions
  \item Bindings introduced: the identifier is bound to the expression
    being matched
  \end{itemize}
\item \hypertarget{wild_pat}{\textbf{Wildcard patterns}}
  \begin{itemize}
  \item Syntax: \texttt{_} (underscore, note this use is completely
    independent of \texttt{_} for \hyperlink{type_inference_sec}{type
      inference})
  \item Type for match: all types
  \item Expressions matched: all expressions
  \item Bindings introduced: none.  Hence it is like a
      variable pattern that uses a fresh identifier.  Using \texttt{_}
      is better style because it indicates the value matched is not
      used.  Notice that ``\texttt{let _ = e;}'' is equivalent to
      \texttt{e}.
  \end{itemize}
  
\item \textbf{Reference patterns}
  \begin{itemize}
  \item Syntax: \texttt{*x} (i.e., the * character followed by an
    identifier)
  \item Types for match: all types
  \item Expressions matched: all expressions.  Note: Currently,
    reference patterns may only appear inside of other patterns so
    that the compiler can determine the region for the pointer type
    assigned to \texttt{x}.
  \item Bindings introduced: \texttt{x} is bound to \emph{the address of}
    the expression being matched.  Hence if matched against a value
    of type \texttt{t} in region \texttt{r}, the type of \texttt{x} is
    \texttt{t@r}. 
  \end{itemize}
  
\item \textbf{Numeric constant patterns}
  \begin{itemize}
  \item Syntax: An \texttt{int}, \texttt{char}, or \texttt{float} constant
  \item Types for match: numeric types
  \item Expressions matched: numeric values such that \texttt{==}
    applied to the value and the pattern yields true.  (Standard C
    numeric promotions apply.  Note that comparing floating point
    values for equality is usually a bad idea.)
  \item Bindings introduced: none
  \end{itemize}
  
\item \textbf{\texttt{null} constant patterns}
  \begin{itemize}
  \item Syntax: \texttt{null}
  \item Types for match: nullable pointer types, including \texttt{?} types
  \item Expressions matched: \texttt{null}
  \item Bindings introduced: none
  \end{itemize}
  
\item \textbf{enum patterns}
  \begin{itemize}
  \item Syntax: an enum constant
  \item Types for match: the enum type containing the constant
  \item Expressions matched: the constant
  \item Bindings introduced: none
  \end{itemize}
  
\item \textbf{Tuple patterns}
  \begin{itemize}
  \item Syntax: \texttt{\$(p1,...,pn)} where p1,...,pn are patterns
  \item Types for match: tuple types where pi matches the type of the
    tuple's ith field \texttt{i} between 1 and \texttt{n}.
  \item Expressions matched: tuples where the ith field matches \texttt{pi} for
    \texttt{i} between 1 and \texttt{n}.
  \item Bindings introduced: bindings introduced by \texttt{p1}, \ldots,
    \texttt{pn}.
  \end{itemize}
  
\item \textbf{Struct patterns}
  \begin{itemize}
  \item Syntax: There are two forms:
    \begin{itemize}
    \item \texttt{X(p1,...,pn)} where \texttt{X} is the name of a struct
      with \texttt{n} fields and p1,...,pn are patterns.  This syntax is
      shorthand for \verb|X{.f1 = p1, ..., .fn = pn}| where fi is the
      ith field in X\@.
    \item \verb|X{.f1 = p1, ..., .fn = pn}| where the fields of X
      are f1, ..., fn but not necessarily in that order
    \end{itemize}
      
  \item Types for match: \texttt{struct X} (or instantiations when
    \texttt{struct X} is polymorphic) such that pi matches the type of
    fi for \texttt{i} between 1 and \texttt{n}.
  \item Expressions matched: structs where the value in fi matches pi
    for \texttt{i} between 1 and \texttt{n}.
  \item Bindings introduced: bindings introduced by p1,...,pn
  \end{itemize}
  
\item \textbf{Pointer patterns}
  \begin{itemize}
  \item Syntax: \texttt{\&p} where p is a pattern
  \item Types for match: pointer types, including \texttt{?} types.
    Also \texttt{tunion Foo} (or instantiations of it) when the pattern
    is \texttt{\&Bar(p1,...,pn)} and \texttt{Bar} is a value-carrying
    variant of \texttt{tunion Foo} and pi matches the type of the ith
    value carried by Bar.
  \item Expressions matched: non-null pointers where the value pointed
    to matches p.  Note this explanation includes the case where the
    expression has type tunion Foo and the pattern is
    \texttt{\&Bar(p1,...,pn)} and the current variant of the expression is
    ``pointer to \texttt{Bar}''.
  \item Bindings introduced: bindings introduced by p
  \end{itemize}
  
\item \textbf{tunion and xtunion patterns}
  \begin{itemize}
  \item Syntax: \texttt{X} if X is a variant that carries no values.
    Else \texttt{X(p1,...,pn)} where X is the name of a variant (that
    has no existential type parameters) and p1, ..., pn are patterns.
    If X has existential type parameters, the syntax is
    \texttt{X<`t1,...,`tm>(p1,...,pn)} for distinct \texttt{`t1}, \ldots,
    \texttt{`tm}.
  \item Types for match: If \texttt{X} is non-value-carrying variant of
    \texttt{tunion Foo}, then types \texttt{tunion Foo} and \texttt{tunion Foo.x}
    (or instantiations of them).  If \texttt{X} carries values, then
    \texttt{tunion Foo.X} (or instantiations of it) where the pi matches
    the type of ith field.  The number of existential type variables in the
    pattern must be the number of existential type variables for
    tunion \texttt{Foo.X}.
  \item Expressions matched: If \texttt{X} is non-value-carrying, then
    \texttt{X}.  If \texttt{X} is value-carrying, then values created from
    the constructor \texttt{X} such that pi matches the ith field.
  \item Bindings introduced: bindings introduced by p1,...,pn
  \end{itemize}
\end{itemize}

\subsection{Switch Statements}\hypertarget{switch_stmt}{}

In Cyclone, you can switch on a value of any type and the case
``labels'' (the part between case and the colon) are patterns.  The
switch expression is evaluated and then matched against each pattern
in turn.  The first matching case statement is executed.  Except for
some restrictions, Cyclone's switch statement is therefore a powerful
extension of C's switch statement.

\paragraph{Restrictions}
\begin{itemize}
\item \emph{You cannot implicitly ``fall-through'' to the next case}.
  Instead, you must use the \texttt{fallthru;} statement, which has
  the effect of transferring control to the beginning of the next
  case.  Even the last case may not implicitly ``fall-through.''
  \texttt{fallthru} is discussed more below.
\item The cases in a switch \emph{must be exhaustive}; it is a
  compile-time error if the compiler determines that it could be that
  no case matches.  The rules for what the compiler determines are
  described below.
\item A \emph{case cannot be unreachable}.  It is a compile-time error
  if the compiler determines that a later case may be subsumed by an
  earlier one.  The rules for what the compiler determines are
  described below.  (C almost has this restriction because case labels
  cannot be repeated, but Cyclone is more restrictive.  For example, C
  allows cases after a default case.)
\item The body of a switch statement must be a \emph{sequence of case
    statements} and case statements can appear only in such a
  sequence. So idioms like Duff's device
  (such as ``\texttt{switch(i\%4) while(i-- >=0) \{ case 3: ... \}}'')
  are not supported.
\item A constant case label must be a constant, \emph{not a constant
    expression}.  That is, case 3+4: is allowed in C, but not in
  Cyclone.
\end{itemize}

\paragraph{An Extension of C}

Except for the above restrictions, we can see Cyclone's switch is an
extension of C's switch.  For example, consider this code (which has
the same meaning in C and Cyclone):

\begin{verbatim}
  int f(int i) {
    switch(i) {
    case 0:  return 17;
    case 1:  return 17;
    default: return i;
    }
  }
\end{verbatim}

In Cyclone terms, the code tries to match against the constant 0.  If
it does not match (i is not 0), it tries to match against the pattern
1.  Everything matches against default; in fact, default is just
alternate notation for ``\texttt{case _}'', i.e., a case with a
\hyperlink{wild_pat}{wildcard pattern}.  For performance reasons,
switch statements that are legal C switch statements are translated to
C switch statements.  Other switch statements are translated to,
``a mess of tests and gotos''.

We now discuss some of the restrictions in terms of the above example.
Because there is no ``implicit fallthrough'', the body of case 0
cannot be empty.  However, we can replace the ``return 17;'' with
``fallthru;'' a special Cyclone statement that immediately transfers
control to the next case.  fallthru does not have to appear at the end
of a case body, so it acts more like a goto than a fallthrough.  As in
our example, any case that matches all values of the type switched
upon (e.g., \texttt{default:}, \texttt{case _:}, \texttt{case x:}) must
appear last, otherwise later cases would be unreachable.  (Note that
other types may have even more such patterns.  For example Pair(x,y)
matches all values of type struct Pair {int x; int y};)

\paragraph{Much More Powerful}

Because Cyclone case labels are patterns, a switch statement can match
against any expression and bind parts of the expression to variables.
Also, \textbf{fallthru can (in fact, must) bind values} to the next
case's pattern variables.  This silly example demonstrates all of
these features:

\begin{verbatim}
  extern int f(int);}
  int g(int x, int y) {
    // return f(x)*f(y), but try to avoid using multiplication
    switch($(f(x),f(y))) {
    case $(0,_): fallthru;
    case $(_,0): return 0;
    case $(1,b): fallthru(b+1-1);
    case $(a,1): return a;
    case $(a,b): return a*b;
    }
  }
\end{verbatim}

The only part of this example using a still-unexplained feature is
``fallthru(b)'', but we explain the full example anyway.  The switch
expression has type \texttt{\$(int,int)}, so all of the cases must
have patterns that match this type.  Legal case forms for this type
not used in the example include ``\texttt{case \$(_,id)}:'',
``\texttt{case \$(id,_):}'',
``\texttt{case id:}'',
``\texttt{case _:}'',
and
``default:''.

The code does the following:

\begin{itemize}
\item It evaluates the pair \texttt{\$(f(x), f(y))} and stores the
  result on the stack.
\item If f(x) returned 0, the first case matches, control jumps to the second
  case, and 0 is returned. 
\item Else if f(y) returned 0, the second case matches and 0 is returned.  
\item Else if f(x) returned 1, the third case matches, b is assigned the value
  f(y) returned, control jumps to the fourth case after assigning b+1-1 to a,
  and a (i.e., b + 1 - 1, i.e., b, i.e., f(y)) is returned.
\item Else if f(y) returned 1, the fourth case matches, a is assigned the value
  f(x) returned, and a is returned.
\item Else the last case matches, a is assigned the value f(x) returned, b is
  assigned the value f(y) returned, and a*b is returned.
\end{itemize}

Note that the switch expression is evaluated only once.
Implementation-wise, the result is stored in a compiler-generated
local variable and the value of this variable is used for the ensuring
pattern matches.

The general form of fallthrus is as follows: If the next case has no
bindings (i.e., identifiers in its pattern), then you must write
\texttt{fallthru;}.  If the next case has n bindings, then you must
write \texttt{fallthru(e1,...,en)} where each ei is an expression with
the appropriate type for the ith binding in the next case's pattern,
reading from left to right.  (By appropriate type, we mean the type of
the expression that would be bound to the ith binding were the next
case to match.)  The effect is to evaluate e1 through en, bind them to
the identifiers, and then goto the body of the next case.
\texttt{fallthru} is not allowed in the last case of a switch, not
even if there is an enclosing switch.

We repeat that fallthru may appear anywhere in a case body, but it is
usually used at the end, where its name makes the most sense.  ML
programmers may notice that fallthru with bindings is strictly more
expressive than or-patterns, but more verbose.

\paragraph{Case Guards}

We have withheld the full form of Cyclone case labels.  In addition to
case p: where p is a pattern, you may write \texttt{case p \&\& e:} where
p is a pattern and e is an expression of type int.  (And since
\texttt{e1 \&\& e2} is an expression, you can write
\texttt{case p \&\& e1 \&\& e2:} and so on.)  Let's call e the case's
\textit{guard}.


The case matches if p matches the expression in the switch and e
evaluates to a non-zero value.  e is evaluated only if p matches and
only after the bindings caused by the match have been properly
initialized.  Here is a silly example:

\begin{verbatim}
extern int f(int);
int g(int a, int b) {
  switch ($(a,b-1)) {
  case $(0,y) &amp;&amp; y > 1: return 1;
  case $(3,y) &amp;&amp; f(x+y) == 7 : return 2;
  case $(4,72): return 3;
  default: return 3;
  }
}
\end{verbatim}


The function g returns 1 if a is 0 and b is greater than 2.  Else if x
is 3, it calls the function f (which of course may do arbitrary
things) with the sum of a and b.  If the result is 7, then 2 is
returned.  In all other cases (x is not 3 or the call to f does not
return 7), 3 is returned.


Case guards make constant patterns unnecessary (we can replace case 3:
with \texttt{case x \&\& x==3:}, for example), but constant patterns are
better style and easier to use.


Where clauses are not interpreted by the compiler when doing
exhaustiveness and overlap checks, as explained below.


\paragraph{Exhaustiveness and Useless-Case Checking}

As mentioned before, it is a compile-time error for the type of the
switch expression to have values that none of the case patterns match
or for a pattern not to match any values that earlier patterns do not
already match.  Rather than explain the precise rules, we currently
rely on your intuition.  But there are two rules to guide your
intuition:

\begin{itemize}
\item In terms of exhaustiveness checking, the compiler acts as if any
  case guard might evaluate to false.
\item In terms of exhaustiveness checking, numeric constants cannot
  make patterns exhaustive.  Even if you list out all 256 characters,
  the compiler will act as though there is another possibility you
  have not checked.
\end{itemize}

We emphasize that checking does not just involve the ``top-level'' of
patterns.  For example, the compiler rejects the switch below because
the third case is redundant:

\begin{verbatim}
  enum Color { Red, Green };
  void f(enum Color c1, enum Color c2) {
    switch ($(c1,c2)) {
    case $(Red,x): return;
    case $(x,Green): return;
    case $(Red,Green): return;
    default: return;
    }
  }
\end{verbatim}

\paragraph{Rules for No Implicit Fall-Through}
As mentioned several times now, Cyclone differs from C in that a case
body may not implicitly fall-through to the next case.  It is a
compile-time error if such a fall-through might occur.  Because the
compiler cannot determine exactly if an implicit fall-through could
occur, it uses a precise set of rules, which we only sketch here.  The
exact same rules are used to ensure that a function (with return type
other than void) does not ``fall off the bottom.''  The rules
are very similar to the rules for ensuring that Java methods do not
``fall off the bottom.''

The general intuition is that there must be a break, continue, goto,
return, or throw along all control-flow paths.  The value of
expressions is not considered except for numeric constants and logical
combinations (using \texttt{\&\&}, \texttt{||}, and \texttt{?} \texttt{:}) of
such constants.  The statement try s catch \ldots{} is checked as
though an exception might be thrown at any point while s executes.

% Local Variables:
% TeX-master: "main-screen"
% End:
