\section{Varargs}

C functions that take a variable number of arguments (vararg
functions) are syntactically convenient for the caller, but C makes it
very difficult to ensure safety.  The callee has no fool-proof way to
determine the number of arguments or even their types.  Also, there is
no type information for the compiler to use at call-sites to reject
bad calls.

Cyclone provides three styles of vararg functions that provide
different trade-offs for safety, efficiency, and convenience.

First, you can call C vararg functions just as you would in C:
\begin{verbatim}
extern "C" void foo(int x, ...);
void g() { 
  foo(3, 7, "hi", 'x');
}
\end{verbatim}
However, for the reasons described above, \texttt{foo} is almost
surely unsafe.  All the Cyclone compiler will do is ensure that the
vararg arguments at the call site have some legal Cyclone type.

Actually, you can declare a Cyclone function to take C-style varargs,
but Cyclone provides no way to access the vararg arguments for this
style.  That is why the example refers to a C function.  (In the
future, function subtyping could make this style less than completely
silly for Cyclone functions.)

The second style is for a variable number of arguments of one type:
\begin{verbatim}
void foo(int x, ...string_t args);
void g() {
  foo(17, "hi", "mom");
}
\end{verbatim}
The syntax is a type and identifer after the ``\texttt{...}''.  (The
identifier is optional in prototypes, as with other parameters.)  You
can use any identifier; \texttt{args} is not special.  At the
call-site, Cyclone will ensure that each vararg has the correct type,
in this case \texttt{string_t}.

Accessing the varargs is simpler than in C.  Continuing our example,
\texttt{args} has type \texttt{string_t *@fat `foo} in the body of
\texttt{foo}.  You retrieve the first argument (\texttt{"hi"}) with
\texttt{args[0]}, the second argument (\texttt{"mom"}) with
\texttt{args[1]}, and so on.  Of course, \texttt{numelts(args)} tells you
how many arguments there are.

This style is implemented as follows: At the call-site, the compiler
generates a stack-allocated array with the array elements.  It then
passes a ``fat pointer'' to the callee with bounds indicating the
number of elements in the array.  Compared to C-style varargs, this
style is less efficient because there is a bounds-check and an extra
level of indirection for each vararg access.  But we get safety and
using vararg functions is just as convenient.  No heap allocation
occurs.

A useful example of this style is in the list library:
\begin{verbatim}
list_t<`a> list(... `a argv) {
  list_t result = NULL;
  for (int i = numelts(argv) - 1; i >= 0; i--) 
    result = new List{argv[i],result};
  return result;
}
\end{verbatim}
Callers can now write \texttt{list(1,2,3,4,5)} and get a list of 5
elements.

The third style addresses the problem that it's often desirable to
have a function take a variable number of arguments of
\emph{different} types.  For example, \texttt{printf} works this way.
In Cyclone, we could use a \texttt{datatype} in conjunction with the
second style.  The callee then uses an array subscript to access a
vararg and a switch statement to determine its \texttt{datatype}
variant.  But this would not be very convenient for the caller---it
would have to explicitly ``wrap'' each vararg in the \texttt{datatype}
type.  The third style makes this wrapping implicit.  For example, the
type of printf in Cyclone is:

\begin{verbatim}
extern datatype PrintArg<`r::R> {
  String_pa(const char ? *@notnull @nozeroterm`r);
  Int_pa(unsigned long);
  Double_pa(double);
  LongDouble_pa(long double);
  ShortPtr_pa(short *@notnull `r);
  IntPtr_pa(unsigned long *@notnull `r);
};
typedef datatype PrintArg<`r> *@notnull `r parg_t<`r>;
int printf(const char *@fat fmt, ... inject parg_t);
\end{verbatim}

The special syntax ``\texttt{inject}'' is the syntactic distinction
for the third style.  The type must be a \texttt{datatype} type.  In the
body of the vararg function, the array holding the vararg elements have
this \texttt{datatype} type, with the function's region.  (That is, the
wrappers are stack-allocated just as the vararg array is.)

At the call-site, the compiler implicitly wraps each vararg by finding
a \texttt{datatype} variant that has the expression's type and using
it.  The exact rules for finding the variant are as follows:  Look in
order for a variant that carries exactly the type of the expression.
Use the first variant that matches.  If none, make a second pass and
find the first variant that carries a type to which the expression can
be coerced.  If none, it is a compile-time error.

In practice, the \texttt{datatype} types used for this style of vararg
tend to be quite specialized and used only for vararg purposes.

Compared to the other styles, the third style is less efficient
because the caller must wrap and the callee unwrap each argument.  But
everything is allocated on the stack and call sites do everything
implicitly.  A testament to the style's power is the library's
implementation of printf and scanf entirely in Cyclone (except for the
actual I/O system calls, of course).

